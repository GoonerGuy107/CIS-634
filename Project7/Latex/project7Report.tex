%% ---------------------------------------------------
%% Kyle Peppe
%% CIS 634 Project 7 Report
%% ---------------------------------------------------
\documentclass{report}

\title{Project 7 Report}
\author{Kyle Peppe}
\date{\today}

%% ---------------------------------------------------
%% 634format specifies the format of our reports
%% ---------------------------------------------------
\usepackage{634format}

%% ---------------------------------------------------
%% enumerate 
%% ---------------------------------------------------
\usepackage{enumerate}

%% ---------------------------------------------------
%% listings is used for including our source code in reports
%% textcomp provides additional symbols
%% ---------------------------------------------------
\usepackage{listings}
\usepackage{textcomp}

%% ---------------------------------------------------
%% Packages for math environments
%% ---------------------------------------------------
\usepackage{amsmath}

%% ---------------------------------------------------
%% Packages for URLs and hotlinks in the table of contents
%% and symbolic cross references using \ref
%% ---------------------------------------------------
\usepackage{hyperref}

%% ---------------------------------------------------
%% Packages for using HOL-generated macros and displays
%% ---------------------------------------------------
\usepackage{holtex}
\usepackage{holtexbasic}
% =====================================================================
%
% Macros for typesetting the HOL system manual
%
% =====================================================================

% ---------------------------------------------------------------------
% Abbreviations for words and phrases
% ---------------------------------------------------------------------

\newcommand\TUTORIAL{{\footnotesize\sl TUTORIAL}}
\newcommand\DESCRIPTION{{\footnotesize\sl DESCRIPTION}}
\newcommand\REFERENCE{{\footnotesize\sl REFERENCE}}
\newcommand\LOGIC{{\footnotesize\sl LOGIC}}
\newcommand\LIBRARIES{{\footnotesize\sl LIBRARIES}}
\usepackage{textcomp}

\newcommand{\bs}{\texttt{\char'134}} % backslash
\newcommand{\lb}{\texttt{\char'173}} % left brace
\newcommand{\rb}{\texttt{\char'175}} % right brace
\newcommand{\td}{\texttt{\char'176}} % tilde
\newcommand{\lt}{\texttt{\char'74}} % less than
\newcommand{\gt}{\texttt{\char'76}} % greater than
\newcommand{\dol}{\texttt{\char'44}} % dollar
\newcommand{\pipe}{\texttt{\char'174}}
\newcommand{\apost}{\texttt{\textquotesingle}}
% double back quotes ``
\newcommand{\dq}{\texttt{\char'140\char'140}}
%These macros were included by slind:

\newcommand{\holquote}[1]{\dq#1\dq}

\def\HOL{\textsc{Hol}}
\def\holn{\HOL}  % i.e. hol n(inety-eight), no digits in
                 % macro names is a bit of a pain; deciding to do away
                 % with hol98 nomenclature means that we just want to
                 % write HOL for hol98.
\def\holnversion{Kananaskis-11}
\def\holnsversion{Kananaskis~11} % version with space rather than hyphen
\def\LCF{{\small LCF}}
\def\LCFLSM{{\small LCF{\kern-.2em}{\normalsize\_}{\kern0.1em}LSM}}
\def\PPL{{\small PP}{\kern-.095em}$\lambda$}
\def\PPLAMBDA{{\small PPLAMBDA}}
\def\ML{{\small ML}}
\def\holmake{\texttt{Holmake}}

\newcommand\ie{\mbox{\textit{i{.}e{.}}}}
\newcommand\eg{\mbox{\textit{e{.}g{.}}}}
\newcommand\viz{\mbox{viz{.}}}
\newcommand\adhoc{\mbox{\it ad hoc}}
\newcommand\etal{{\it et al.\/}}
% NOTE: \etc produces wrong spacing if used between sentences, that is
% like here \etc End such sentences with non-macro etc.
\newcommand\etc{\mbox{\textit{etc{.}}}}

% ---------------------------------------------------------------------
% Simple abbreviations and macros for mathematical typesetting
% ---------------------------------------------------------------------

\newcommand\fun{{\to}}
\newcommand\prd{{\times}}

\newcommand\conj{\ \wedge\ }
\newcommand\disj{\ \vee\ }
\newcommand\imp{ \Rightarrow }
\newcommand\eqv{\ \equiv\ }
\newcommand\cond{\rightarrow}
\newcommand\vbar{\mid}
\newcommand\turn{\ \vdash\ } % FIXME: "\ " results in extra space
\newcommand\hilbert{\varepsilon}
\newcommand\eqdef{\ \equiv\ }

\newcommand\natnums{\mbox{${\sf N}\!\!\!\!{\sf N}$}}
\newcommand\bools{\mbox{${\sf T}\!\!\!\!{\sf T}$}}

\newcommand\p{$\prime$}
\newcommand\f{$\forall$\ }
\newcommand\e{$\exists$\ }

\newcommand\orr{$\vee$\ }
\newcommand\negg{$\neg$\ }

\newcommand\arrr{$\rightarrow$}
\newcommand\hex{$\sharp $}

\newcommand{\uquant}[1]{\forall #1.\ }
\newcommand{\equant}[1]{\exists #1.\ }
\newcommand{\hquant}[1]{\hilbert #1.\ }
\newcommand{\iquant}[1]{\exists ! #1.\ }
\newcommand{\lquant}[1]{\lambda #1.\ }

\newcommand{\leave}[1]{\\[#1]\noindent}
\newcommand\entails{\mbox{\rule{.3mm}{4mm}\rule[2mm]{.2in}{.3mm}}}

% ---------------------------------------------------------------------
% Font-changing commands
% ---------------------------------------------------------------------

\newcommand{\theory}[1]{\hbox{{\small\tt #1}}}
\newcommand{\theoryimp}[1]{\texttt{#1}}

\newcommand{\con}[1]{{\sf #1}}
\newcommand{\rul}[1]{{\tt #1}}
\newcommand{\ty}[1]{\textsl{#1}}

\newcommand{\ml}[1]{\mbox{{\def\_{\char'137}\texttt{#1}}}}
\newcommand{\holtxt}[1]{\ml{#1}}
\newcommand\ms{\tt}
\newcommand{\s}[1]{{\small #1}}

\newcommand{\pin}[1]{{\bf #1}}
% FIXME: for multichar symbols \mathit should be used.
\def\m#1{\mbox{\normalsize$#1$}}

% ---------------------------------------------------------------------
% Abbreviations for particular mathematical constants etc.
% ---------------------------------------------------------------------

\newcommand\T{\con{T}}
\newcommand\F{\con{F}}
\newcommand\OneOne{\con{One\_One}}
\newcommand\OntoSubset{\con{Onto\_Subset}}
\newcommand\Onto{\con{Onto}}
\newcommand\TyDef{\con{Type\_Definition}}
\newcommand\Inv{\con{Inv}}
\newcommand\com{\con{o}}
\newcommand\Id{\con{I}}
\newcommand\MkPair{\con{Mk\_Pair}}
\newcommand\IsPair{\con{Is\_Pair}}
\newcommand\Fst{\con{Fst}}
\newcommand\Snd{\con{Snd}}
\newcommand\Suc{\con{Suc}}
\newcommand\Nil{\con{Nil}}
\newcommand\Cons{\con{Cons}}
\newcommand\Hd{\con{Hd}}
\newcommand\Tl{\con{Tl}}
\newcommand\Null{\con{Null}}
\newcommand\ListPrimRec{\con{List\_Prim\_Rec}}


\newcommand\SimpRec{\con{Simp\_Rec}}
\newcommand\SimpRecRel{\con{Simp\_Rec\_Rel}}
\newcommand\SimpRecFun{\con{Simp\_Rec\_Fun}}
\newcommand\PrimRec{\con{Prim\_Rec}}
\newcommand\PrimRecRel{\con{Prim\_Rec\_Rel}}
\newcommand\PrimRecFun{\con{Prim\_Rec\_Fun}}

\newcommand\bool{\ty{bool}}
\newcommand\num{\ty{num}}
\newcommand\ind{\ty{ind}}
\newcommand\lst{\ty{list}}

% ---------------------------------------------------------------------
% \minipagewidth = \textwidth minus 1.02 em
% ---------------------------------------------------------------------

\newlength{\minipagewidth}
\setlength{\minipagewidth}{\textwidth}
\addtolength{\minipagewidth}{-1.02em}

% ---------------------------------------------------------------------
% Environment for the items on the title page of a case study
% ---------------------------------------------------------------------

\newenvironment{inset}[1]{\noindent{\large\bf #1}\begin{list}%
{}{\setlength{\leftmargin}{\parindent}%
\setlength{\topsep}{-.1in}}\item }{\end{list}\vskip .4in}

% ---------------------------------------------------------------------
% Macros for little HOL sessions displayed in boxes.
%
% Usage: (1) \setcounter{sessioncount}{1} resets the session counter
%
%        (2) \begin{session}\begin{verbatim}
%             .
%              < lines from hol session >
%             .
%            \end{verbatim}\end{session}
%
%            typesets the session in a numbered box.
% ---------------------------------------------------------------------

\newlength{\hsbw}
\setlength{\hsbw}{\textwidth}
\addtolength{\hsbw}{-\arrayrulewidth}
\addtolength{\hsbw}{-\tabcolsep}
\newcommand\HOLSpacing{13pt}

\newcounter{sessioncount}
\setcounter{sessioncount}{0}

\newenvironment{session}{\begin{flushleft}
 \refstepcounter{sessioncount}
 \begin{tabular}{@{}|c@{}|@{}}\hline
 \begin{minipage}[b]{\hsbw}
 \vspace*{-.5pt}
 \begin{flushright}
 \rule{0.01in}{.15in}\rule{0.3in}{0.01in}\hspace{-0.35in}
 \raisebox{0.04in}{\makebox[0.3in][c]{\footnotesize\sl \thesessioncount}}
 \end{flushright}
 \vspace*{-.55in}
 \begingroup\small\baselineskip\HOLSpacing}{\endgroup\end{minipage}\\ \hline
 \end{tabular}
 \end{flushleft}}

% ---------------------------------------------------------------------
% Macro for boxed ML functions, etc.
%
% Usage: (1) \begin{holboxed}\begin{verbatim}
%               .
%               < lines giving names and types of mk functions >
%               .
%            \end{verbatim}\end{holboxed}
%
%            typesets the given lines in a box.
%
%            Conventions: lines are left-aligned under the "g" of begin,
%            and used to highlight primary reference for the ml function(s)
%            that appear in the box.
% ---------------------------------------------------------------------

\newenvironment{holboxed}{\begin{flushleft}
  \begin{tabular}{@{}|c@{}|@{}}\hline
  \begin{minipage}[b]{\hsbw}
% \vspace*{-.55in}
  \vspace*{.06in}
  \begingroup\small\baselineskip\HOLSpacing}{\endgroup\end{minipage}\\ \hline
  \end{tabular}
  \end{flushleft}}

% ---------------------------------------------------------------------
% Macro for unboxed ML functions, etc.
%
% Usage: (1) \begin{hol}\begin{verbatim}
%               .
%               < lines giving names and types of mk functions >
%               .
%            \end{verbatim}\end{hol}
%
%            typesets the given lines exactly like {boxed}, except there's
%            no box.
%
%            Conventions: lines are left-aligned under the "g" of begin,
%            and used to display ML code in verbatim, left aligned.
% ---------------------------------------------------------------------

\newenvironment{hol}{\begin{flushleft}
 \begin{tabular}{c@{}@{}}
 \begin{minipage}[b]{\hsbw}
% \vspace*{-.55in}
 \vspace*{.06in}
 \begingroup\small\baselineskip\HOLSpacing}{\endgroup\end{minipage}\\
 \end{tabular}
 \end{flushleft}}

% ---------------------------------------------------------------------
% Emphatic brackets
% ---------------------------------------------------------------------

\newcommand\leb{\lbrack\!\lbrack}
\newcommand\reb{\rbrack\!\rbrack}


% ---------------------------------------------------------------------
% Quotations
% ---------------------------------------------------------------------


%These macros were included by ap; they are used in Chapters 9 and 10
%of the HOL DESCRIPTION

\newcommand{\inds}%standard infinite set
 {\mbox{\rm I}}

\newcommand{\ch}%standard choice function
 {\mbox{\rm ch}}

\newcommand{\den}[1]%denotational brackets
 {[\![#1]\!]}

\newcommand{\two}%standard 2-element set
 {\mbox{\rm 2}}


\begin{document}

%% --------------------------------------------------- 
%% The listings  parameter "language" is set to "ML"
%% ---------------------------------------------------
\lstset{language=ML}


\maketitle{}

\begin{abstract}
  This project demonstrated my ability to use the properties of
  symmetric and asymmetric key encryption and decryption to prove
  theroems using CipherTheory. I also proved a theory using the
  properties of signature verification. I have completed the exercises
  15.6.1, 15.6.2, and 15.6.3.  This project includes the following
  packages:
	\begin{description}
		\item[\emph{634format.sty}] A format style for this course
		\item[\emph{listings}] Package for displaying and inputting ML source code
		\item[\emph{holtex}] HOL style files and commands to display in the report
	\end{description}
        This document also demonstrates my ability to :
	\begin{itemize}
		\item Easily generate a table of contents,
		\item Refer to chapter and section labels
	\end{itemize}
\end{abstract}

\tableofcontents{}

\begin{acknowledgments}
  I would like to acknowledge the 2 professors, Professor Chin and 
  Professor Hamner, that have helped me begin to understand this new
  ML programming language. Also to Syracuse University for accepting
  me to this Masters program in Cybersecurity.
\end{acknowledgments}

\chapter{Executive Summary}
\label{cha:executive-summary}
\textbf{All requirements for this project are satisfied.}
Specifically,
\begin{description}
\item[Report Contents] \ \\
  Our report has the following content:
  \begin{enumerate}[{}]
  \item Chapter~\ref{cha:executive-summary}: Executive Summary
  \item Chapter~\ref{cha:15-6-1}: Exercise 15.6.1
    \begin{enumerate}[{}]
    \item Section~\ref{problem-statement-15-6-1}: Problem Statement
    \item Section~\ref{proof-15-6-1-A}: Proof of exercise15_6_1a_thm
    \item Section~\ref{proof-15-6-1-B}: Proof of exercise15_6_1b_thm
    \end{enumerate}
  \item Chapter~\ref{cha:15-6-2}: Exercise 15.6.2
    \begin{enumerate}[{}]
    \item Section~\ref{problem-statement-15-6-2}: Problem Statement
    \item Section~\ref{proof-15-6-2-A}: Proof of exercise15_6_2a_thm
    \item Section~\ref{proof-15-6-2-B}: Proof of exercise15 6_2b_thm
    \end{enumerate}
  \item Chapter~\ref{cha:15-6-3}: Exercise 15.6.3
    \begin{enumerate}[{}]
    \item Section~\ref{problem-statement-15-6-3}: Problem Statement
    \item Section~\ref{rel-15-6-3}: Relevant Code
    \item Section~\ref{trans-15-6-3}: Transcript of Proof
    \end{enumerate}
 \item Appendix~\ref{cha:source-code-ciph}: Source Code for cipherScript.sml
 \item Appendix~\ref{cha:source-code-crypto}: Source Code for cryptoExercisesScript.sml
  \end{enumerate}
\item[Reproducibility in ML and \LaTeX{}] \ \\
  The ML and \LaTeX{} source files compile with no errors.
\end{description}


\chapter{Excercise 15.6.1}
\label{cha:15-6-1}

\section{Problem statement}
\label{problem-statement-15-6-1}
Use the properties of symmetric key encryption and decryption in
cipherTheory to prove the following 2 theroems:

A) exercise15_6_1a_thm:
!key enMsg message.(deciphS key enMsg = SOME message) <=>
    (enMsg = Es key (SOME message))
B) exercise15_6_1a_thm:
!keyAlice k text.(deciphS keyAlice (Es k (SOME text)) 
    = SOME "This is from Alice")<=> ((k = keyAlice) /\ 
    (text = "This is from Alice"))

\section{Proof of exercise15_6_1a_thm}
\label{proof-15-6-1-A}
\begin{lstlisting}[frame=TBlr]
val exercise15_6_1a_thm = 
TAC_PROOF(
    ([],``!key enMsg message.(deciphS key enMsg = SOME message) <=>
    (enMsg = Es key (SOME message))``),
    PROVE_TAC[deciphS_one_one]
);
\end{lstlisting}

\lstinputlisting[linerange={2090-2103}]{../HOL/cryptoExercisesScript.trans}

\section{Proof of exercise15_6_1b_thm}
\label{proof-15-6-1-B}
\begin{lstlisting}[frame=TBlr]
val exercise15_6_1b_thm =
TAC_PROOF(
    ([],``!keyAlice k text.(deciphS keyAlice (Es k (SOME text)) 
    = SOME "This is from Alice")<=> ((k = keyAlice) /\ 
    (text = "This is from Alice"))``),
    PROVE_TAC[deciphS_one_one]
);
\end{lstlisting}

\lstinputlisting[linerange={2104-2117}]{../HOL/cryptoExercisesScript.trans}

\chapter{Excercise 15.6.2}
\label{cha:15-6-2}

\section{Problem statement}
\label{problem-statement-15-6-2}
Use the properties of asymmetric key encryption and decryption
to prove the following 2 theroems:

A) exercise15_6_2a_thm:
!P message.(deciphP (pubK P) enMsg = SOME message) <=>
    (enMsg = Ea (privK P) (SOME message))``
B) exercise15_6_2a_thm:
!key text.(deciphP (pubK Alice) (Ea key (SOME text)) 
    = SOME "This is from Alice")<=> (key = privK Alice) 
    /\ (text = "This is from Alice")

\section{Proof of exercise15_6_2a_thm}
\label{proof-15-6-2-A}
\begin{lstlisting}[frame=TBlr]
val exercise15_6_2a_thm =
TAC_PROOF(
    ([],``!P message.(deciphP (pubK P) enMsg = SOME message) <=>
    (enMsg = Ea (privK P) (SOME message))``),
    PROVE_TAC[deciphP_one_one]
);
\end{lstlisting}

\lstinputlisting[linerange={2118-2131}]{../HOL/cryptoExercisesScript.trans}

\section{Proof of exercise15_6_2b_thm}
\label{proof-15-6-2-B}
\begin{lstlisting}[frame=TBlr]
val exercise15_6_2b_thm =
TAC_PROOF(
    ([],``!key text.(deciphP (pubK Alice) (Ea key (SOME text)) 
    = SOME "This is from Alice")<=> (key = privK Alice) 
    /\ (text = "This is from Alice")``),
    PROVE_TAC[deciphP_one_one]
);
\end{lstlisting}

\lstinputlisting[linerange={2132-2146}]{../HOL/cryptoExercisesScript.trans}

\chapter{Excercise 15.6.3}
\label{cha:15-6-3}

\section{Problem statement}
\label{problem-statement-15-6-3}
Use the properties of signature verification to prove the following theorem:

!signature.signVerify (pubK Alice) signature (SOME "This is from Alice")
    <=> (signature = sign (privK Alice) (hash (SOME "This is from Alice")))

\section{Relevant Code}
\label{rel-15-6-3}
\begin{lstlisting}[frame=TBlr]
val exercise15_6_3_thm =
TAC_PROOF(
    ([],``!signature.signVerify (pubK Alice) signature 
    (SOME "This is from Alice") <=> (signature = sign (privK Alice) 
    (hash (SOME "This is from Alice")))``),
    PROVE_TAC[signVerify_one_one]
);
\end{lstlisting}

\section{Transcript of Proof}
\label{trans-15-6-3}
\lstinputlisting[linerange={2147-2161}]{../HOL/cryptoExercisesScript.trans}

\appendix{}
\chapter{Source Code for cipherScript.sml}
\label{cha:source-code-ciph}
\lstinputlisting{../HOL/cipherScript.sml}

\chapter{Source Code for cryptoExercisesScript.sml}
\label{cha:source-code-crypto}
\lstinputlisting{../HOL/cryptoExercisesScript.sml}

\end{document}