%% ---------------------------------------------------
%% Kyle Peppe
%% CIS 634 Project 5 Report
%% ---------------------------------------------------
\documentclass{report}

\title{Project 5 Report}
\author{Kyle Peppe}
\date{\today}

%% ---------------------------------------------------
%% 634format specifies the format of our reports
%% ---------------------------------------------------
\usepackage{634format}

%% ---------------------------------------------------
%% enumerate 
%% ---------------------------------------------------
\usepackage{enumerate}

%% ---------------------------------------------------
%% listings is used for including our source code in reports
%% textcomp provides additional symbols
%% ---------------------------------------------------
\usepackage{listings}
\usepackage{textcomp}

%% ---------------------------------------------------
%% Packages for math environments
%% ---------------------------------------------------
\usepackage{amsmath}

%% ---------------------------------------------------
%% Packages for URLs and hotlinks in the table of contents
%% and symbolic cross references using \ref
%% ---------------------------------------------------
\usepackage{hyperref}

%% ---------------------------------------------------
%% Packages for using HOL-generated macros and displays
%% ---------------------------------------------------
\usepackage{holtex}
\usepackage{holtexbasic}
\input{commands}

\begin{document}

%% --------------------------------------------------- 
%% The listings  parameter "language" is set to "ML"
%% ---------------------------------------------------
\lstset{language=ML}


\maketitle{}

\begin{abstract}
  This project was to demonstrate my command of using the new
  tacticals we have learned in order to prove theorems. This was also
  the first set of exercises where I had to load and open HOL saved
  theory files. The exercises became more difficult with chapter 10
  being a bit more difficult. I have completed the exercises 11.6.1,
  11.6.2, and 11.6.3.
	\begin{itemize}
		\item Problem Statement
		\item Relevant Code
		\item Test Results
	\end{itemize}
        This project includes the following packages:
	\begin{description}
		\item[\emph{634format.sty}] A format style for this course
		\item[\emph{listings}] Package for displaying and inputting ML source code
		\item[\emph{holtex}] HOL style files and commands to display in the report
	\end{description}
        This document also demonstrates my ability to :
	\begin{itemize}
		\item Easily generate a table of contents,
		\item Refer to chapter and section labels
	\end{itemize}
\end{abstract}

\tableofcontents{}

\begin{acknowledgments}
  I would like to acknowledge the 2 professors, Professor Chin and 
  Professor Hamner, that have helped me begin to understand this new
  ML programming language. Also to Syracuse University for accepting
  me to this Masters program in Cybersecurity.
\end{acknowledgments}

\chapter{Executive Summary}
\label{cha:executive-summary}
\textbf{All requirements for this project are satisfied.}
Specifically,
\begin{description}
\item[Report Contents] \ \\
  Our report has the following content:
  \begin{enumerate}[{}]
  \item Chapter~\ref{cha:executive-summary}: Executive Summary
  \item Chapter~\ref{cha:11-6-1}: Exercise 11.6.1
    \begin{enumerate}[{}]
    \item Section~\ref{problem-statement-11-6-1}: Problem Statement
    \item Section~\ref{rel-code-11-6-1}: Relevant Code 
    \item Section~\ref{trans-11-6-1}: Session Transcripts
    \end{enumerate}
  \item Chapter~\ref{cha:11-6-2}: Exercise 11.6.2
    \begin{enumerate}[{}]
    \item Section~\ref{problem-statement-11-6-2}: Problem Statement
    \item Section~\ref{rel-code-11-6-2}: Relevant Code
    \item Section~\ref{trans-11-6-2}: Session Transcripts
    \end{enumerate}
  \item Chapter~\ref{cha:11-6-3}: Exercise 11.6.3
    \begin{enumerate}[{}]
    \item Section~\ref{problem-statement-11-6-3}: Problem Statement
    \item Section~\ref{rel-code-11-6-3}: Relevant Code
    \item Section~\ref{trans-11-6-3}: Session Transcripts
    \end{enumerate}
 \item Appendix~\ref{cha:source-code}: Source Code
  \end{enumerate}
\item[Reproducibility in ML and \LaTeX{}] \ \\
  The ML and \LaTeX{} source files compile with no errors.
\end{description}


\chapter{Excercise 11.6.1}
\label{cha:11-6-1}

\section{Problem statement}
\label{problem-statement-11-6-1}
Add to the theory exTypeTheory by proving the following theorem
\begin{lstlisting}[frame=tblr]
 ``!(l1:'a list)(l2:'a list).LENGTH (APP l1 l2) = LENGTH l1 + LENGTH l2``)
\end{lstlisting}

\section{Relevant Code}
\label{rel-code-11-6-1}
\begin{lstlisting}[frame=TBlr]
(* Exercise 11.6.1			*)
val LENGTH_APP =
TAC_PROOF(
	([], ``!(l1:'a list)(l2:'a list).LENGTH (APP l1 l2)
	= LENGTH l1 + LENGTH l2``),
	Induct_on `l1` THEN
	ASM_REWRITE_TAC [APP_def, LENGTH, ADD_CLAUSES]
);

(* Save the Theorem			*)
val _ = save_thm("LENGTH_APP", LENGTH_APP);

\end{lstlisting}

\section{Session Transcript}
\label{trans-11-6-1}
\begin{session}
  \begin{scriptsize}
\begin{verbatim}
> val LENGTH_APP =
TAC_PROOF(
	([], ``!(l1:'a list)(l2:'a list).LENGTH (APP l1 l2)
	= LENGTH l1 + LENGTH l2``),
	Induct_on `l1` THEN
	ASM_REWRITE_TAC [APP_def, LENGTH, ADD_CLAUSES]
);
# # # # # # val LENGTH_APP =
   [oracles: DISK_THM] [axioms: ] []
|- !(l1 :'a list) (l2 :'a list).
     LENGTH (APP l1 l2) = LENGTH l1 + LENGTH l2:
   thm
> 

\end{verbatim}
  \end{scriptsize}
\end{session}

\chapter{Excercise 11.6.2}
\label{cha:11-6-2}

\section{Problem statement}
\label{problem-statement-11-6-2}
Add to the theory exTypeTheory by defining the function Map (Map f
[1;2;3;4] = [f1; f2; f3; f4] and then prove the following theorem
\begin{lstlisting}[frame=tblr]
``Map f(APP l1 l2) = APP (Map f l1) (Map f l2)``)
\end{lstlisting}

\section{Relevant Code}
\label{rel-code-11-6-2}
\begin{lstlisting}[frame=TBlr]
(* Exercise 11.6.2			*)
val Map_def =
Define
	`(Map (f:'a -> 'b) [] = []) /\ (Map f (h::(l1:'a list))
	= (f h)::Map f (l1:'a list))`;


val Map_APP =
TAC_PROOF (
	  ([], ``Map f(APP l1 l2) = APP (Map f l1) (Map f l2)``),
	  Induct_on `l1` THEN
	  ASM_REWRITE_TAC[Map_def, APP_def]
);

(* Save the Theorem			*)
val _ = save_thm("Map_APP", Map_APP);
\end{lstlisting}

\section{Session Transcript}
\label{trans-11-6-2}
\begin{session}
  \begin{scriptsize}
\begin{verbatim}
> val Map_def =
Define
	`(Map (f:'a -> 'b) [] = []) /\ (Map f (h::(l1:'a list))
	= (f h)::Map f (l1:'a list))`;
# # # Definition has been stored under "Map_def"
val Map_def =
   [oracles: DISK_THM] [axioms: ] []
|- (!(f :'a -> 'b). Map f ([] :'a list) = ([] :'b list)) /\
   !(f :'a -> 'b) (h :'a) (l1 :'a list). Map f (h::l1) = f h::Map f l1:
   thm
> val Map_APP =
TAC_PROOF (
	  ([], ``Map f(APP l1 l2) = APP (Map f l1) (Map f l2)``),
	  Induct_on `l1` THEN
	  ASM_REWRITE_TAC[Map_def, APP_def]
);
# # # # # <<HOL message: inventing new type variable names: 'a, 'b>>
val Map_APP =
   [oracles: DISK_THM] [axioms: ] []
|- Map (f :'b -> 'a) (APP (l1 :'b list) (l2 :'b list)) =
   APP (Map f l1) (Map f l2):
   thm
> 
\end{verbatim}
  \end{scriptsize}
\end{session}

\chapter{Excercise 11.6.3}
\label{cha:11-6-3}

\section{Problem statement}
\label{problem-statement-11-6-3}
Solve the below theorems after using/setting up the new datatypes,
definitions, and theorems: datatype nexp, definition of semantics of
nexp experssions.
\begin{lstlisting}[frame=tblr]
Add_0:
 ``!(f:nexp).nexpVal (Add (Num 0) f) = nexpVal f``

Add_SYM:
``!(f1:nexp)(f2:nexp).nexpVal (Add f1 f2) = nexpVal (Add f2 f1)``

Sub_0:
``!(f:nexp).(nexpVal (Sub (Num 0) f) = 0) /\ 
   (nexpVal (Sub f (Num 0)) = nexpVal f)``

Mult_ASSOC:
``!(f1:nexp)(f2:nexp)(f3:nexp).nexpVal (Mult f1 (Mult f2 f3)) =
	nexpVal (Mult (Mult f1 f2) f3)``
\end{lstlisting}

\section{Relevant Code}
\label{rel-code-11-6-3}
\begin{lstlisting}[frame=TBlr]
(* Add_0   	    	     	   	*)
val Add_0 =
TAC_PROOF(
	([], ``!(f:nexp).nexpVal (Add (Num 0) f) = nexpVal f``),
	Induct_on `f` THEN
	ASM_REWRITE_TAC[ADD_CLAUSES, nexpValDef]
);

(* Save the Theorem			*)
val _ = save_thm("Add_0", Add_0);

(* Add_SYM				*)
val Add_SYM =
TAC_PROOF(
	([], ``!(f1:nexp)(f2:nexp).nexpVal (Add f1 f2) = nexpVal (Add f2 f1)``),
	Induct_on `f1` THEN
	PROVE_TAC[nexpValDef, ADD_SYM]
);

(* Save the Theorem			*)
val _ = save_thm("Add_SYM", Add_SYM);

(* Sub_0				*)
val Sub_0 =
TAC_PROOF(
	([],``!(f:nexp).(nexpVal (Sub (Num 0) f) = 0) /\
	(nexpVal (Sub f (Num 0)) = nexpVal f)``),
	Induct_on`f` THEN
	PROVE_TAC[nexpValDef, SUB_0]
);

(* Save the Theorem			*)
val _ = save_thm("Sub_0", Sub_0);

(* Mult_ASSOC				*)
val Mult_ASSOC =
TAC_PROOF(
	([],``!(f1:nexp)(f2:nexp)(f3:nexp).nexpVal (Mult f1 (Mult f2 f3)) =
	nexpVal (Mult (Mult f1 f2) f3)``),
	Induct_on`f1` THEN
	ASM_REWRITE_TAC[MULT_ASSOC, nexpValDef]
);

(* Save the Theorem			*)
val _ = save_thm("Mult_ASSOC", Mult_ASSOC);
\end{lstlisting}

\section{Session Transcripts}
\label{trans-11-6-3}
\begin{session}
  \begin{scriptsize}
\begin{verbatim}
> val Add_0 =
TAC_PROOF(
	([], ``!(f:nexp).nexpVal (Add (Num 0) f) = nexpVal f``),
	Induct_on `f` THEN
	ASM_REWRITE_TAC[ADD_CLAUSES, nexpValDef]
);
# # # # # val Add_0 =
    [] |- !(f :nexp). nexpVal (Add (Num (0 :num)) f) = nexpVal f:
   thm
> val Add_SYM =
TAC_PROOF(
	([], ``!(f1:nexp)(f2:nexp).nexpVal (Add f1 f2) = nexpVal (Add f2 f1)``),
	Induct_on `f1` THEN
	PROVE_TAC[nexpValDef, ADD_SYM]
);
# # # # # Meson search level: ..........
Meson search level: ..........
Meson search level: ..........
Meson search level: ..........
val Add_SYM =
    [] |- !(f1 :nexp) (f2 :nexp). nexpVal (Add f1 f2) = nexpVal (Add f2 f1):
   thm
> val Sub_0 =
TAC_PROOF(
	([],``!(f:nexp).(nexpVal (Sub (Num 0) f) = 0) /\
	(nexpVal (Sub f (Num 0)) = nexpVal f)``),
	Induct_on`f` THEN
	PROVE_TAC[nexpValDef, SUB_0]
);
# # # # # # Meson search level: ........................
Meson search level: ........................
Meson search level: ........................
Meson search level: ........................
val Sub_0 =
    []
|- !(f :nexp).
     (nexpVal (Sub (Num (0 :num)) f) = (0 :num)) /\
     (nexpVal (Sub f (Num (0 :num))) = nexpVal f):
   thm
> val Mult_ASSOC =
TAC_PROOF(
	([],``!(f1:nexp)(f2:nexp)(f3:nexp).nexpVal (Mult f1 (Mult f2 f3)) =
	nexpVal (Mult (Mult f1 f2) f3)``),
	Induct_on`f1` THEN
	ASM_REWRITE_TAC[MULT_ASSOC, nexpValDef]
);
# # # # # # val Mult_ASSOC =
    []
|- !(f1 :nexp) (f2 :nexp) (f3 :nexp).
     nexpVal (Mult f1 (Mult f2 f3)) = nexpVal (Mult (Mult f1 f2) f3):
   thm
> 
\end{verbatim}
  \end{scriptsize}
\end{session}

\appendix{}

\chapter{Source code}
\label{cha:source-code}
\lstinputlisting{../exTypeScript.sml}

\lstinputlisting{../nexpScript.sml}

\end{document}