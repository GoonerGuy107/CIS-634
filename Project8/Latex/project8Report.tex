%% ---------------------------------------------------
%% Kyle Peppe
%% CIS 634 Project 8 Report
%% ---------------------------------------------------
\documentclass{report}

\title{Project 8 Report}
\author{Kyle Peppe}
\date{\today}

%% ---------------------------------------------------
%% 634format specifies the format of our reports
%% ---------------------------------------------------
\usepackage{634format}

%% ---------------------------------------------------
%% enumerate 
%% ---------------------------------------------------
\usepackage{enumerate}

%% ---------------------------------------------------
%% listings is used for including our source code in reports
%% textcomp provides additional symbols
%% ---------------------------------------------------
\usepackage{listings}
\usepackage{textcomp}

%% ---------------------------------------------------
%% Packages for math environments
%% ---------------------------------------------------
\usepackage{amsmath}

%% ---------------------------------------------------
%% Packages for URLs and hotlinks in the table of contents
%% and symbolic cross references using \ref
%% ---------------------------------------------------
\usepackage{hyperref}

%% ---------------------------------------------------
%% Packages for using HOL-generated macros and displays
%% ---------------------------------------------------
\usepackage{holtex}
\usepackage{holtexbasic}
\input{commands}

\begin{document}

%% --------------------------------------------------- 
%% The listings  parameter "language" is set to "ML"
%% ---------------------------------------------------
\lstset{language=ML}


\maketitle{}

\begin{abstract}
  This project demonstrated my ability to create machine states,
  defining datatypes, and using the previous 2 things to prove the
  theroems provided for the many theorems for the problems 16.3.1 and
  16.3.2. This project includes the following packages:
	\begin{description}
		\item[\emph{634format.sty}] A format style for this course
		\item[\emph{listings}] Package for displaying and inputting ML source code
		\item[\emph{holtex}] HOL style files and commands to display in the report
	\end{description}
        This document also demonstrates my ability to :
	\begin{itemize}
		\item Easily generate a table of contents,
		\item Refer to chapter and section labels
	\end{itemize}
\end{abstract}

\tableofcontents{}

\begin{acknowledgments}
  I would like to acknowledge the 2 professors, Professor Chin and 
  Professor Hamner, that have helped me begin to understand this new
  ML programming language. Also to Syracuse University for accepting
  me to this Masters program in Cybersecurity.
\end{acknowledgments}

\chapter{Executive Summary}
\label{cha:executive-summary}
\textbf{All requirements for this project are satisfied.}
Specifically,
\begin{description}
\item[Report Contents] \ \\
  Our report has the following content:
  \begin{enumerate}[{}]
  \item Chapter~\ref{cha:executive-summary}: Executive Summary
  \item Chapter~\ref{cha:16-3-1}: Exercise 16.3.1
    \begin{enumerate}[{}]
    \item Section~\ref{problem-statement-16-3-1}: Problem Statement
    \item Section~\ref{defs-16-3-1-A}: 16-3-1A: Definitions and Theorems of Datatypes
    \item Section~\ref{defs-16-3-1-B}: 16-3-1B: Definitions of M1ns and M1out
    \item Section~\ref{proofs-16-3-1-C}: 16-3-1C: Proofs of m1TR_rules, m1TR_clauses, m1Trans_Equiv_TR, and m1_rules
    \end{enumerate}
  \item Chapter~\ref{cha:16-3-2}: Exercise 16.3.2
    \begin{enumerate}[{}]
    \item Section~\ref{problem-statement-16-3-2}: Problem Statement
    \item Section~\ref{defs-16-3-2-A}: 16-3-2A: Definitions and Theorems of Datatypes 
    \item Section~\ref{defs-16-3-2-B}: 16-3-2B: Definitions of ctrNS_def and ctrOut_def
    \item Section~\ref{proofs-16-3-2-C}: 16-3-2C: Proofs of ctrTR_rules, ctrTR_clauses, ctrTrans_Equiv_TR, and ctr_rules
    \end{enumerate}
 \item Appendix~\ref{cha:source-code-sm}: Source Code for smScript.sml
 \item Appendix~\ref{cha:source-code-m1}: Source Code for m1Script.sml
 \item Appendix~\ref{cha:source-code-counter}: Source Code for counterScript.sml
  \end{enumerate}
\item[Reproducibility in ML and \LaTeX{}] \ \\
  The ML and \LaTeX{} source files compile with no errors.
\end{description}


\chapter{Excercise 16.3.1}
\label{cha:16-3-1}

\section{Problem statement}
\label{problem-statement-16-3-1}
This problem had 3 different sections, in the first section I had to
define datatypes for the inputs, states and outputs. For the 2nd part
I defined the next state and output functions, followed by proving 4
theories about the m1.

\section{16-3-1A: Definitions and Theorems of Datatypes}
\label{defs-16-3-1-A}
\lstinputlisting[linerange={15-30}]{../HOL/M1/m1Script.sml}

\lstinputlisting[linerange={520-537}]{../HOL/M1/m1Script.trans}

\section{16-3-1B: Definitions of M1ns and M1out}
\label{defs-16-3-1-B}
\lstinputlisting[linerange={32-42}]{../HOL/M1/m1Script.sml}

\lstinputlisting[linerange={538-565}]{../HOL/M1/m1Script.trans}

\section{16-3-1C: Proofs of m1TR_rules, m1TR_clauses, m1Trans_Equiv_TR, and m1_rules}
\label{proofs-16-3-1-C}
\lstinputlisting[linerange={44-70}]{../HOL/M1/m1Script.sml}

\lstinputlisting[linerange={566-639}]{../HOL/M1/m1Script.trans}

\chapter{Exercise 16.3.2}
\label{cha:16-3-2}

\section{Problem statement}
\label{problem-statement-16-3-2}
This problem had 3 different sections, in the first section I had to
define datatypes for the inputs, states and outputs. For the 2nd part
I defined the next state and output functions, followed by proving 4
theories about the counter.

\section{16-3-2A: Definitions and Theorems of Datatypes}
\label{defs-16-3-2-A}
\lstinputlisting[linerange={15-30}]{../HOL/Counter/counterScript.sml}

\lstinputlisting[linerange={520-538}]{../HOL/Counter/counterScript.trans}

\section{16-3-2B: Definitions of ctrNS_def and ctrOut_def}
\label{defs-16-3-2-B}
\lstinputlisting[linerange={32-42}]{../HOL/Counter/counterScript.sml}

\lstinputlisting[linerange={539-570}]{../HOL/Counter/counterScript.trans}

\section{16-3-2C: Proofs of ctrTR_rules, ctrTR_clauses, ctrTrans_Equiv_TR, and ctr_rules}
\label{proofs-16-3-2-C}
\lstinputlisting[linerange={44-66}]{../HOL/Counter/counterScript.sml}

\lstinputlisting[linerange={571-630}]{../HOL/Counter/counterScript.trans}

\appendix{}

\chapter{Source Code for smScript.sml}
\label{cha:source-code-sm}
\lstinputlisting{../HOL/smScript.sml}

\chapter{Source Code for m1Script.sml}
\label{cha:source-code-m1}
\lstinputlisting{../HOL/M1/m1Script.sml}

\chapter{Source Code for counterScript.sml}
\label{cha:source-code-counter}
\lstinputlisting{../HOL/Counter/counterScript.sml}

\end{document}