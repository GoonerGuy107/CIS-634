%% ---------------------------------------------------
%% Kyle Peppe
%% CIS 634 Project 2 Report
%% ---------------------------------------------------
\documentclass{report}

\title{Project 1 Report}
\author{Kyle Peppe}
\date{\today}

%% ---------------------------------------------------
%% 634format specifies the format of our reports
%% ---------------------------------------------------
\usepackage{634format}

%% ---------------------------------------------------
%% enumerate 
%% ---------------------------------------------------
\usepackage{enumerate}

%% ---------------------------------------------------
%% listings is used for including our source code in reports
%% textcomp provides additional symbols
%% ---------------------------------------------------
\usepackage{listings}
\usepackage{textcomp}

%% ---------------------------------------------------
%% Packages for math environments
%% ---------------------------------------------------
\usepackage{amsmath}

%% ---------------------------------------------------
%% Packages for URLs and hotlinks in the table of contents
%% and symbolic cross references using \ref
%% ---------------------------------------------------
\usepackage{hyperref}

%% ---------------------------------------------------
%% Packages for using HOL-generated macros and displays
%% ---------------------------------------------------
\usepackage{holtex}
\usepackage{holtexbasic}
% =====================================================================
%
% Macros for typesetting the HOL system manual
%
% =====================================================================

% ---------------------------------------------------------------------
% Abbreviations for words and phrases
% ---------------------------------------------------------------------

\newcommand\TUTORIAL{{\footnotesize\sl TUTORIAL}}
\newcommand\DESCRIPTION{{\footnotesize\sl DESCRIPTION}}
\newcommand\REFERENCE{{\footnotesize\sl REFERENCE}}
\newcommand\LOGIC{{\footnotesize\sl LOGIC}}
\newcommand\LIBRARIES{{\footnotesize\sl LIBRARIES}}
\usepackage{textcomp}

\newcommand{\bs}{\texttt{\char'134}} % backslash
\newcommand{\lb}{\texttt{\char'173}} % left brace
\newcommand{\rb}{\texttt{\char'175}} % right brace
\newcommand{\td}{\texttt{\char'176}} % tilde
\newcommand{\lt}{\texttt{\char'74}} % less than
\newcommand{\gt}{\texttt{\char'76}} % greater than
\newcommand{\dol}{\texttt{\char'44}} % dollar
\newcommand{\pipe}{\texttt{\char'174}}
\newcommand{\apost}{\texttt{\textquotesingle}}
% double back quotes ``
\newcommand{\dq}{\texttt{\char'140\char'140}}
%These macros were included by slind:

\newcommand{\holquote}[1]{\dq#1\dq}

\def\HOL{\textsc{Hol}}
\def\holn{\HOL}  % i.e. hol n(inety-eight), no digits in
                 % macro names is a bit of a pain; deciding to do away
                 % with hol98 nomenclature means that we just want to
                 % write HOL for hol98.
\def\holnversion{Kananaskis-11}
\def\holnsversion{Kananaskis~11} % version with space rather than hyphen
\def\LCF{{\small LCF}}
\def\LCFLSM{{\small LCF{\kern-.2em}{\normalsize\_}{\kern0.1em}LSM}}
\def\PPL{{\small PP}{\kern-.095em}$\lambda$}
\def\PPLAMBDA{{\small PPLAMBDA}}
\def\ML{{\small ML}}
\def\holmake{\texttt{Holmake}}

\newcommand\ie{\mbox{\textit{i{.}e{.}}}}
\newcommand\eg{\mbox{\textit{e{.}g{.}}}}
\newcommand\viz{\mbox{viz{.}}}
\newcommand\adhoc{\mbox{\it ad hoc}}
\newcommand\etal{{\it et al.\/}}
% NOTE: \etc produces wrong spacing if used between sentences, that is
% like here \etc End such sentences with non-macro etc.
\newcommand\etc{\mbox{\textit{etc{.}}}}

% ---------------------------------------------------------------------
% Simple abbreviations and macros for mathematical typesetting
% ---------------------------------------------------------------------

\newcommand\fun{{\to}}
\newcommand\prd{{\times}}

\newcommand\conj{\ \wedge\ }
\newcommand\disj{\ \vee\ }
\newcommand\imp{ \Rightarrow }
\newcommand\eqv{\ \equiv\ }
\newcommand\cond{\rightarrow}
\newcommand\vbar{\mid}
\newcommand\turn{\ \vdash\ } % FIXME: "\ " results in extra space
\newcommand\hilbert{\varepsilon}
\newcommand\eqdef{\ \equiv\ }

\newcommand\natnums{\mbox{${\sf N}\!\!\!\!{\sf N}$}}
\newcommand\bools{\mbox{${\sf T}\!\!\!\!{\sf T}$}}

\newcommand\p{$\prime$}
\newcommand\f{$\forall$\ }
\newcommand\e{$\exists$\ }

\newcommand\orr{$\vee$\ }
\newcommand\negg{$\neg$\ }

\newcommand\arrr{$\rightarrow$}
\newcommand\hex{$\sharp $}

\newcommand{\uquant}[1]{\forall #1.\ }
\newcommand{\equant}[1]{\exists #1.\ }
\newcommand{\hquant}[1]{\hilbert #1.\ }
\newcommand{\iquant}[1]{\exists ! #1.\ }
\newcommand{\lquant}[1]{\lambda #1.\ }

\newcommand{\leave}[1]{\\[#1]\noindent}
\newcommand\entails{\mbox{\rule{.3mm}{4mm}\rule[2mm]{.2in}{.3mm}}}

% ---------------------------------------------------------------------
% Font-changing commands
% ---------------------------------------------------------------------

\newcommand{\theory}[1]{\hbox{{\small\tt #1}}}
\newcommand{\theoryimp}[1]{\texttt{#1}}

\newcommand{\con}[1]{{\sf #1}}
\newcommand{\rul}[1]{{\tt #1}}
\newcommand{\ty}[1]{\textsl{#1}}

\newcommand{\ml}[1]{\mbox{{\def\_{\char'137}\texttt{#1}}}}
\newcommand{\holtxt}[1]{\ml{#1}}
\newcommand\ms{\tt}
\newcommand{\s}[1]{{\small #1}}

\newcommand{\pin}[1]{{\bf #1}}
% FIXME: for multichar symbols \mathit should be used.
\def\m#1{\mbox{\normalsize$#1$}}

% ---------------------------------------------------------------------
% Abbreviations for particular mathematical constants etc.
% ---------------------------------------------------------------------

\newcommand\T{\con{T}}
\newcommand\F{\con{F}}
\newcommand\OneOne{\con{One\_One}}
\newcommand\OntoSubset{\con{Onto\_Subset}}
\newcommand\Onto{\con{Onto}}
\newcommand\TyDef{\con{Type\_Definition}}
\newcommand\Inv{\con{Inv}}
\newcommand\com{\con{o}}
\newcommand\Id{\con{I}}
\newcommand\MkPair{\con{Mk\_Pair}}
\newcommand\IsPair{\con{Is\_Pair}}
\newcommand\Fst{\con{Fst}}
\newcommand\Snd{\con{Snd}}
\newcommand\Suc{\con{Suc}}
\newcommand\Nil{\con{Nil}}
\newcommand\Cons{\con{Cons}}
\newcommand\Hd{\con{Hd}}
\newcommand\Tl{\con{Tl}}
\newcommand\Null{\con{Null}}
\newcommand\ListPrimRec{\con{List\_Prim\_Rec}}


\newcommand\SimpRec{\con{Simp\_Rec}}
\newcommand\SimpRecRel{\con{Simp\_Rec\_Rel}}
\newcommand\SimpRecFun{\con{Simp\_Rec\_Fun}}
\newcommand\PrimRec{\con{Prim\_Rec}}
\newcommand\PrimRecRel{\con{Prim\_Rec\_Rel}}
\newcommand\PrimRecFun{\con{Prim\_Rec\_Fun}}

\newcommand\bool{\ty{bool}}
\newcommand\num{\ty{num}}
\newcommand\ind{\ty{ind}}
\newcommand\lst{\ty{list}}

% ---------------------------------------------------------------------
% \minipagewidth = \textwidth minus 1.02 em
% ---------------------------------------------------------------------

\newlength{\minipagewidth}
\setlength{\minipagewidth}{\textwidth}
\addtolength{\minipagewidth}{-1.02em}

% ---------------------------------------------------------------------
% Environment for the items on the title page of a case study
% ---------------------------------------------------------------------

\newenvironment{inset}[1]{\noindent{\large\bf #1}\begin{list}%
{}{\setlength{\leftmargin}{\parindent}%
\setlength{\topsep}{-.1in}}\item }{\end{list}\vskip .4in}

% ---------------------------------------------------------------------
% Macros for little HOL sessions displayed in boxes.
%
% Usage: (1) \setcounter{sessioncount}{1} resets the session counter
%
%        (2) \begin{session}\begin{verbatim}
%             .
%              < lines from hol session >
%             .
%            \end{verbatim}\end{session}
%
%            typesets the session in a numbered box.
% ---------------------------------------------------------------------

\newlength{\hsbw}
\setlength{\hsbw}{\textwidth}
\addtolength{\hsbw}{-\arrayrulewidth}
\addtolength{\hsbw}{-\tabcolsep}
\newcommand\HOLSpacing{13pt}

\newcounter{sessioncount}
\setcounter{sessioncount}{0}

\newenvironment{session}{\begin{flushleft}
 \refstepcounter{sessioncount}
 \begin{tabular}{@{}|c@{}|@{}}\hline
 \begin{minipage}[b]{\hsbw}
 \vspace*{-.5pt}
 \begin{flushright}
 \rule{0.01in}{.15in}\rule{0.3in}{0.01in}\hspace{-0.35in}
 \raisebox{0.04in}{\makebox[0.3in][c]{\footnotesize\sl \thesessioncount}}
 \end{flushright}
 \vspace*{-.55in}
 \begingroup\small\baselineskip\HOLSpacing}{\endgroup\end{minipage}\\ \hline
 \end{tabular}
 \end{flushleft}}

% ---------------------------------------------------------------------
% Macro for boxed ML functions, etc.
%
% Usage: (1) \begin{holboxed}\begin{verbatim}
%               .
%               < lines giving names and types of mk functions >
%               .
%            \end{verbatim}\end{holboxed}
%
%            typesets the given lines in a box.
%
%            Conventions: lines are left-aligned under the "g" of begin,
%            and used to highlight primary reference for the ml function(s)
%            that appear in the box.
% ---------------------------------------------------------------------

\newenvironment{holboxed}{\begin{flushleft}
  \begin{tabular}{@{}|c@{}|@{}}\hline
  \begin{minipage}[b]{\hsbw}
% \vspace*{-.55in}
  \vspace*{.06in}
  \begingroup\small\baselineskip\HOLSpacing}{\endgroup\end{minipage}\\ \hline
  \end{tabular}
  \end{flushleft}}

% ---------------------------------------------------------------------
% Macro for unboxed ML functions, etc.
%
% Usage: (1) \begin{hol}\begin{verbatim}
%               .
%               < lines giving names and types of mk functions >
%               .
%            \end{verbatim}\end{hol}
%
%            typesets the given lines exactly like {boxed}, except there's
%            no box.
%
%            Conventions: lines are left-aligned under the "g" of begin,
%            and used to display ML code in verbatim, left aligned.
% ---------------------------------------------------------------------

\newenvironment{hol}{\begin{flushleft}
 \begin{tabular}{c@{}@{}}
 \begin{minipage}[b]{\hsbw}
% \vspace*{-.55in}
 \vspace*{.06in}
 \begingroup\small\baselineskip\HOLSpacing}{\endgroup\end{minipage}\\
 \end{tabular}
 \end{flushleft}}

% ---------------------------------------------------------------------
% Emphatic brackets
% ---------------------------------------------------------------------

\newcommand\leb{\lbrack\!\lbrack}
\newcommand\reb{\rbrack\!\rbrack}


% ---------------------------------------------------------------------
% Quotations
% ---------------------------------------------------------------------


%These macros were included by ap; they are used in Chapters 9 and 10
%of the HOL DESCRIPTION

\newcommand{\inds}%standard infinite set
 {\mbox{\rm I}}

\newcommand{\ch}%standard choice function
 {\mbox{\rm ch}}

\newcommand{\den}[1]%denotational brackets
 {[\![#1]\!]}

\newcommand{\two}%standard 2-element set
 {\mbox{\rm 2}}


\begin{document}

%% --------------------------------------------------- 
%% The listings  parameter "language" is set to "ML"
%% ---------------------------------------------------
\lstset{language=ML}


\maketitle{}

\begin{abstract}
  This project built on our initial learning of the ML programming
  language and some more complicated problems that we had to
  solve. This project finished up with us performing some basic HOL
  instructions. Then I put the findings, code and output into Latex to
  further enhance my knowledge on making these reports. I did the
  exercises from the PDF book from the class 8.4.1, 8.4.2, and
  8.4.3. Below are the sections that are in this report for each
  problem:
	\begin{itemize}
		\item Problem Statement
		\item Relevant Code
		\item Test Results
	\end{itemize}
        This project includes the following packages:
	\begin{description}
		\item[\emph{634format.sty}] A format style for this course
		\item[\emph{listings}] Package for displaying and inputting ML source code
		\item[\emph{holtex}] HOL style files and commands to display in the report
	\end{description}
        This document also demonstrates my ability to :
	\begin{itemize}
		\item Easily generate a table of contents,
		\item Refer to chapter and section labels
	\end{itemize}.
\end{abstract}

\tableofcontents{}

\begin{acknowledgments}
  I would like to acknowledge the 2 professors, Professor Chin and 
  Professor Hamner, that have helped me begin to understand this new
  ML programming language. Also to Syracuse University for accepting
  me to this Masters program in Cybersecurity.
\end{acknowledgments}

\chapter{Executive Summary}
\label{cha:executive-summary}
\textbf{All requirements for this project are satisfied.}
Specifically,
\begin{description}
\item[Report Contents] \ \\
  Our report has the following content:
  \begin{enumerate}[{}]
  \item Chapter~\ref{cha:executive-summary}: Executive Summary
  \item Chapter~\ref{cha:8.4.1}: Exercise 8.4.1
    \begin{enumerate}[{}]
    \item Section~\ref{problem-statement-8-4-1}: Problem Statement
    \item Section~\ref{rel-code-8-4-1}: Relevant Code 
    \item Section~\ref{trans-8-4-1}: Session Transcripts
    \end{enumerate}
  \item Chapter~\ref{cha:8.4.2}: Exercise 8.4.2
    \begin{enumerate}[{}]
    \item Section~\ref{problem-statement-8-4-2}: Problem Statement
    \item Section~\ref{rel-code-8-4-2}: Relevant Code
    \item Section~\ref{trans-8-4-2}: Session Transcripts
    \end{enumerate}
  \item Chapter~\ref{cha:8.4.3}: Exercise 8.4.3
    \begin{enumerate}[{}]
    \item Section~\ref{problem-statement-8-4-3}: Problem Statement
    \item Section~\ref{rel-code-8-4-3}: Relevant Code
    \item Section~\ref{trans-8-4-3}: Session Transcripts
    \end{enumerate}
 \item Appendix~\ref{cha:source-code}: Source Code
  \end{enumerate}
\item[Reproducibility in ML and \LaTeX{}] \ \\
  The ML and \LaTeX{} source files compile with no errors.
\end{description}


\chapter{Excercise 8.4.1}
\label{cha:8.4.1}

\section{Problem statement}
\label{problem-statement-8-4-1}
Prove the following theorem
\begin{lstlisting}[frame=tblr]
> val problem1Thm =
	[] |- p ==> (p ==> q) ==> (q ==> r) ==> r
	: thm
\end{lstlisting}

\section{Relevant Code}
\label{rel-code-8-4-1}
\begin{lstlisting}[frame=TBlr]
val problem1Thm =
let
	val th1 = ASSUME ``p:bool``
	val th2 = ASSUME ``p ==> q``
	val th3 = ASSUME ``q ==> r``
	val th4 = MP th2 th1
	val th5 = MP th3 th4
	val t1 = hd(hyp th1)
	val t2 = hd(hyp th2)
	val t3 = hd(hyp th3)
	val th6 = DISCH t3 th4
	val th7 = DISCH t2 th6
in
	DISCH t3 th7
end

val _ = save_thm("problem1Thm", problem1Thm);

\end{lstlisting}

\section{Session Transcript}
\label{trans-8-4-1}
\begin{session}
  \begin{scriptsize}
\begin{verbatim}
---------------------------------------------------------------------
       HOL-4 [Kananaskis 11 (stdknl, built Sat Aug 19 09:30:06 2017)]

       For introductory HOL help, type: help "hol";
       To exit type <Control>-D
---------------------------------------------------------------------
> > > > # # # # # # # # # ** types trace now on
> # # # # # # # # # ** Unicode trace now off
> val problem1Thm =
let
	val th1 = ASSUME ``p:bool``
	val th2 = ASSUME ``p ==> q``
	val th3 = ASSUME ``q ==> r``
	val th4 = MP th2 th1
	val th5 = MP th3 th4
	val t1 = hd(hyp th1)
	val t2 = hd(hyp th2)
	val t3 = hd(hyp th3)
	val th6 = DISCH t3 th4
	val th7 = DISCH t2 th6
in
	DISCH t3 th7
end;
# # # # # # # # # # # # # # val problem1Thm =
    [.]
|- ((q :bool) ==> (r :bool)) ==> ((p :bool) ==> q) ==> (q ==> r) ==> q:
   thm
> 
\end{verbatim}
  \end{scriptsize}
\end{session}

\chapter{Excercise 8.4.2}
\label{cha:8.4.2}

\section{Problem statement}
\label{problem-statement-8-4-2}
Prove the following theorem:
\begin{lstlisting}[frame=tblr]
> val conjSymThm =
	[] |- p /\ q <=> q /\ p
	: thm
\end{lstlisting}

\section{Relevant Code}
\label{rel-code-8-4-2}
\begin{lstlisting}[frame=TBlr]
val conjSymThm =
let
	val th1 = ASSUME ``p /\ q``
	val th2 = ASSUME ``q /\ p``
	val ab = CONJUNCT1 th1
	val cd = CONJUNCT2 th1
	val ba = CONJUNCT2 th2
	val bc = CONJUNCT1 th2
	val th3 = CONJ cd ab
	val th4 = CONJ ba bc
	val t1 = hd(hyp th1)
	val t2 = hd(hyp th2)
	val th5 = DISCH t1 th3
	val th6 = DISCH t2 th4
in
	IMP_ANTISYM_RULE th5 th6
end

val _ = save_thm("conjSymThm", conjSymThm);
\end{lstlisting}

\section{Session Transcript}
\label{trans-8-4-2}
\begin{session}
  \begin{scriptsize}
\begin{verbatim}
---------------------------------------------------------------------
       HOL-4 [Kananaskis 11 (stdknl, built Sat Aug 19 09:30:06 2017)]

       For introductory HOL help, type: help "hol";
       To exit type <Control>-D
---------------------------------------------------------------------
> > > > # # # # # # # # # ** types trace now on
> # # # # # # # # # ** Unicode trace now off
> val conjSymThm =
let
	val th1 = ASSUME ``p /\ q``
	val th2 = ASSUME ``q /\ p``
	val ab = CONJUNCT1 th1
	val cd = CONJUNCT2 th1
	val ba = CONJUNCT2 th2
	val bc = CONJUNCT1 th2
	val th3 = CONJ cd ab
	val th4 = CONJ ba bc
	val t1 = hd(hyp th1)
	val t2 = hd(hyp th2)
	val th5 = DISCH t1 th3
	val th6 = DISCH t2 th4
in
	IMP_ANTISYM_RULE th5 th6
end;
# # # # # # # # # # # # # # # # val conjSymThm =
   |- (p :bool) /\ (q :bool) <=> q /\ p:
   thm
> 
\end{verbatim}
  \end{scriptsize}
\end{session}

\chapter{Excercise 8.4.3}
\label{cha:8.4.3}

\section{Problem statement}
\label{problem-statement-8-4-3}
Extend your proof in Problem 2 by one step and prove:
\begin{lstlisting}[frame=tblr]
> val conjSymThmAll =
	[] |- !p q. p /\ q <=> q /\ p
	: thm
\end{lstlisting}

\section{Relevant Code}
\label{rel-code-8-4-3}
\begin{lstlisting}[frame=TBlr]
val conjSymThmAll = GENL [``p:bool``, ``q:bool``] conjSymThm;

val _ = save_thm("conjSymThmAll", conjSymThmAll);
\end{lstlisting}

\section{Test Case}
\label{trans-8-4-3}
\begin{session}
  \begin{scriptsize}
\begin{verbatim}
---------------------------------------------------------------------
       HOL-4 [Kananaskis 11 (stdknl, built Sat Aug 19 09:30:06 2017)]

       For introductory HOL help, type: help "hol";
       To exit type <Control>-D
---------------------------------------------------------------------
> > > > # # # # # # # # # ** types trace now on
> # # # # # # # # # ** Unicode trace now off
> val conjSymThm =
let
	val th1 = ASSUME ``p /\ q``
	val th2 = ASSUME ``q /\ p``
	val ab = CONJUNCT1 th1
	val cd = CONJUNCT2 th1
	val ba = CONJUNCT2 th2
	val bc = CONJUNCT1 th2
	val th3 = CONJ cd ab
	val th4 = CONJ ba bc
	val t1 = hd(hyp th1)
	val t2 = hd(hyp th2)
	val th5 = DISCH t1 th3
	val th6 = DISCH t2 th4
in
	IMP_ANTISYM_RULE th5 th6
end;
# # # # # # # # # # # # # # # # val conjSymThm =
   |- (p :bool) /\ (q :bool) <=> q /\ p:
   thm
> val conjSymThmAll = GENL [``p:bool``, ``q:bool``] conjSymThm;

val _ = save_thm("conjSymThmAll", conjSymThmAll);
val conjSymThmAll =
   |- !(p :bool) (q :bool). p /\ q <=> q /\ p:
   thm
> > > 
\end{verbatim}
  \end{scriptsize}
\end{session}


\appendix{}

\chapter{Source code}
\label{cha:source-code}
\lstinputlisting{../chapter8Script.sml}

\end{document}