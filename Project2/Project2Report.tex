%% ---------------------------------------------------
%% Kyle Peppe
%% CIS 634 Project 2 Report
%% ---------------------------------------------------
\documentclass{report}

\title{Project 1 Report}
\author{Kyle Peppe}
\date{\today}

%% ---------------------------------------------------
%% 634format specifies the format of our reports
%% ---------------------------------------------------
\usepackage{634format}

%% ---------------------------------------------------
%% enumerate 
%% ---------------------------------------------------
\usepackage{enumerate}

%% ---------------------------------------------------
%% listings is used for including our source code in reports
%% textcomp provides additional symbols
%% ---------------------------------------------------
\usepackage{listings}
\usepackage{textcomp}

%% ---------------------------------------------------
%% Packages for math environments
%% ---------------------------------------------------
\usepackage{amsmath}

%% ---------------------------------------------------
%% Packages for URLs and hotlinks in the table of contents
%% and symbolic cross references using \ref
%% ---------------------------------------------------
\usepackage{hyperref}

%% ---------------------------------------------------
%% Packages for using HOL-generated macros and displays
%% ---------------------------------------------------
\usepackage{holtex}
\usepackage{holtexbasic}
% =====================================================================
%
% Macros for typesetting the HOL system manual
%
% =====================================================================

% ---------------------------------------------------------------------
% Abbreviations for words and phrases
% ---------------------------------------------------------------------

\newcommand\TUTORIAL{{\footnotesize\sl TUTORIAL}}
\newcommand\DESCRIPTION{{\footnotesize\sl DESCRIPTION}}
\newcommand\REFERENCE{{\footnotesize\sl REFERENCE}}
\newcommand\LOGIC{{\footnotesize\sl LOGIC}}
\newcommand\LIBRARIES{{\footnotesize\sl LIBRARIES}}
\usepackage{textcomp}

\newcommand{\bs}{\texttt{\char'134}} % backslash
\newcommand{\lb}{\texttt{\char'173}} % left brace
\newcommand{\rb}{\texttt{\char'175}} % right brace
\newcommand{\td}{\texttt{\char'176}} % tilde
\newcommand{\lt}{\texttt{\char'74}} % less than
\newcommand{\gt}{\texttt{\char'76}} % greater than
\newcommand{\dol}{\texttt{\char'44}} % dollar
\newcommand{\pipe}{\texttt{\char'174}}
\newcommand{\apost}{\texttt{\textquotesingle}}
% double back quotes ``
\newcommand{\dq}{\texttt{\char'140\char'140}}
%These macros were included by slind:

\newcommand{\holquote}[1]{\dq#1\dq}

\def\HOL{\textsc{Hol}}
\def\holn{\HOL}  % i.e. hol n(inety-eight), no digits in
                 % macro names is a bit of a pain; deciding to do away
                 % with hol98 nomenclature means that we just want to
                 % write HOL for hol98.
\def\holnversion{Kananaskis-11}
\def\holnsversion{Kananaskis~11} % version with space rather than hyphen
\def\LCF{{\small LCF}}
\def\LCFLSM{{\small LCF{\kern-.2em}{\normalsize\_}{\kern0.1em}LSM}}
\def\PPL{{\small PP}{\kern-.095em}$\lambda$}
\def\PPLAMBDA{{\small PPLAMBDA}}
\def\ML{{\small ML}}
\def\holmake{\texttt{Holmake}}

\newcommand\ie{\mbox{\textit{i{.}e{.}}}}
\newcommand\eg{\mbox{\textit{e{.}g{.}}}}
\newcommand\viz{\mbox{viz{.}}}
\newcommand\adhoc{\mbox{\it ad hoc}}
\newcommand\etal{{\it et al.\/}}
% NOTE: \etc produces wrong spacing if used between sentences, that is
% like here \etc End such sentences with non-macro etc.
\newcommand\etc{\mbox{\textit{etc{.}}}}

% ---------------------------------------------------------------------
% Simple abbreviations and macros for mathematical typesetting
% ---------------------------------------------------------------------

\newcommand\fun{{\to}}
\newcommand\prd{{\times}}

\newcommand\conj{\ \wedge\ }
\newcommand\disj{\ \vee\ }
\newcommand\imp{ \Rightarrow }
\newcommand\eqv{\ \equiv\ }
\newcommand\cond{\rightarrow}
\newcommand\vbar{\mid}
\newcommand\turn{\ \vdash\ } % FIXME: "\ " results in extra space
\newcommand\hilbert{\varepsilon}
\newcommand\eqdef{\ \equiv\ }

\newcommand\natnums{\mbox{${\sf N}\!\!\!\!{\sf N}$}}
\newcommand\bools{\mbox{${\sf T}\!\!\!\!{\sf T}$}}

\newcommand\p{$\prime$}
\newcommand\f{$\forall$\ }
\newcommand\e{$\exists$\ }

\newcommand\orr{$\vee$\ }
\newcommand\negg{$\neg$\ }

\newcommand\arrr{$\rightarrow$}
\newcommand\hex{$\sharp $}

\newcommand{\uquant}[1]{\forall #1.\ }
\newcommand{\equant}[1]{\exists #1.\ }
\newcommand{\hquant}[1]{\hilbert #1.\ }
\newcommand{\iquant}[1]{\exists ! #1.\ }
\newcommand{\lquant}[1]{\lambda #1.\ }

\newcommand{\leave}[1]{\\[#1]\noindent}
\newcommand\entails{\mbox{\rule{.3mm}{4mm}\rule[2mm]{.2in}{.3mm}}}

% ---------------------------------------------------------------------
% Font-changing commands
% ---------------------------------------------------------------------

\newcommand{\theory}[1]{\hbox{{\small\tt #1}}}
\newcommand{\theoryimp}[1]{\texttt{#1}}

\newcommand{\con}[1]{{\sf #1}}
\newcommand{\rul}[1]{{\tt #1}}
\newcommand{\ty}[1]{\textsl{#1}}

\newcommand{\ml}[1]{\mbox{{\def\_{\char'137}\texttt{#1}}}}
\newcommand{\holtxt}[1]{\ml{#1}}
\newcommand\ms{\tt}
\newcommand{\s}[1]{{\small #1}}

\newcommand{\pin}[1]{{\bf #1}}
% FIXME: for multichar symbols \mathit should be used.
\def\m#1{\mbox{\normalsize$#1$}}

% ---------------------------------------------------------------------
% Abbreviations for particular mathematical constants etc.
% ---------------------------------------------------------------------

\newcommand\T{\con{T}}
\newcommand\F{\con{F}}
\newcommand\OneOne{\con{One\_One}}
\newcommand\OntoSubset{\con{Onto\_Subset}}
\newcommand\Onto{\con{Onto}}
\newcommand\TyDef{\con{Type\_Definition}}
\newcommand\Inv{\con{Inv}}
\newcommand\com{\con{o}}
\newcommand\Id{\con{I}}
\newcommand\MkPair{\con{Mk\_Pair}}
\newcommand\IsPair{\con{Is\_Pair}}
\newcommand\Fst{\con{Fst}}
\newcommand\Snd{\con{Snd}}
\newcommand\Suc{\con{Suc}}
\newcommand\Nil{\con{Nil}}
\newcommand\Cons{\con{Cons}}
\newcommand\Hd{\con{Hd}}
\newcommand\Tl{\con{Tl}}
\newcommand\Null{\con{Null}}
\newcommand\ListPrimRec{\con{List\_Prim\_Rec}}


\newcommand\SimpRec{\con{Simp\_Rec}}
\newcommand\SimpRecRel{\con{Simp\_Rec\_Rel}}
\newcommand\SimpRecFun{\con{Simp\_Rec\_Fun}}
\newcommand\PrimRec{\con{Prim\_Rec}}
\newcommand\PrimRecRel{\con{Prim\_Rec\_Rel}}
\newcommand\PrimRecFun{\con{Prim\_Rec\_Fun}}

\newcommand\bool{\ty{bool}}
\newcommand\num{\ty{num}}
\newcommand\ind{\ty{ind}}
\newcommand\lst{\ty{list}}

% ---------------------------------------------------------------------
% \minipagewidth = \textwidth minus 1.02 em
% ---------------------------------------------------------------------

\newlength{\minipagewidth}
\setlength{\minipagewidth}{\textwidth}
\addtolength{\minipagewidth}{-1.02em}

% ---------------------------------------------------------------------
% Environment for the items on the title page of a case study
% ---------------------------------------------------------------------

\newenvironment{inset}[1]{\noindent{\large\bf #1}\begin{list}%
{}{\setlength{\leftmargin}{\parindent}%
\setlength{\topsep}{-.1in}}\item }{\end{list}\vskip .4in}

% ---------------------------------------------------------------------
% Macros for little HOL sessions displayed in boxes.
%
% Usage: (1) \setcounter{sessioncount}{1} resets the session counter
%
%        (2) \begin{session}\begin{verbatim}
%             .
%              < lines from hol session >
%             .
%            \end{verbatim}\end{session}
%
%            typesets the session in a numbered box.
% ---------------------------------------------------------------------

\newlength{\hsbw}
\setlength{\hsbw}{\textwidth}
\addtolength{\hsbw}{-\arrayrulewidth}
\addtolength{\hsbw}{-\tabcolsep}
\newcommand\HOLSpacing{13pt}

\newcounter{sessioncount}
\setcounter{sessioncount}{0}

\newenvironment{session}{\begin{flushleft}
 \refstepcounter{sessioncount}
 \begin{tabular}{@{}|c@{}|@{}}\hline
 \begin{minipage}[b]{\hsbw}
 \vspace*{-.5pt}
 \begin{flushright}
 \rule{0.01in}{.15in}\rule{0.3in}{0.01in}\hspace{-0.35in}
 \raisebox{0.04in}{\makebox[0.3in][c]{\footnotesize\sl \thesessioncount}}
 \end{flushright}
 \vspace*{-.55in}
 \begingroup\small\baselineskip\HOLSpacing}{\endgroup\end{minipage}\\ \hline
 \end{tabular}
 \end{flushleft}}

% ---------------------------------------------------------------------
% Macro for boxed ML functions, etc.
%
% Usage: (1) \begin{holboxed}\begin{verbatim}
%               .
%               < lines giving names and types of mk functions >
%               .
%            \end{verbatim}\end{holboxed}
%
%            typesets the given lines in a box.
%
%            Conventions: lines are left-aligned under the "g" of begin,
%            and used to highlight primary reference for the ml function(s)
%            that appear in the box.
% ---------------------------------------------------------------------

\newenvironment{holboxed}{\begin{flushleft}
  \begin{tabular}{@{}|c@{}|@{}}\hline
  \begin{minipage}[b]{\hsbw}
% \vspace*{-.55in}
  \vspace*{.06in}
  \begingroup\small\baselineskip\HOLSpacing}{\endgroup\end{minipage}\\ \hline
  \end{tabular}
  \end{flushleft}}

% ---------------------------------------------------------------------
% Macro for unboxed ML functions, etc.
%
% Usage: (1) \begin{hol}\begin{verbatim}
%               .
%               < lines giving names and types of mk functions >
%               .
%            \end{verbatim}\end{hol}
%
%            typesets the given lines exactly like {boxed}, except there's
%            no box.
%
%            Conventions: lines are left-aligned under the "g" of begin,
%            and used to display ML code in verbatim, left aligned.
% ---------------------------------------------------------------------

\newenvironment{hol}{\begin{flushleft}
 \begin{tabular}{c@{}@{}}
 \begin{minipage}[b]{\hsbw}
% \vspace*{-.55in}
 \vspace*{.06in}
 \begingroup\small\baselineskip\HOLSpacing}{\endgroup\end{minipage}\\
 \end{tabular}
 \end{flushleft}}

% ---------------------------------------------------------------------
% Emphatic brackets
% ---------------------------------------------------------------------

\newcommand\leb{\lbrack\!\lbrack}
\newcommand\reb{\rbrack\!\rbrack}


% ---------------------------------------------------------------------
% Quotations
% ---------------------------------------------------------------------


%These macros were included by ap; they are used in Chapters 9 and 10
%of the HOL DESCRIPTION

\newcommand{\inds}%standard infinite set
 {\mbox{\rm I}}

\newcommand{\ch}%standard choice function
 {\mbox{\rm ch}}

\newcommand{\den}[1]%denotational brackets
 {[\![#1]\!]}

\newcommand{\two}%standard 2-element set
 {\mbox{\rm 2}}


\begin{document}

%% --------------------------------------------------- 
%% The listings  parameter "language" is set to "ML"
%% ---------------------------------------------------
\lstset{language=ML}


\maketitle{}

\begin{abstract}
  This project built on our initial learning of the ML programming
  language and some more complicated problems that we had to
  solve. This project finished up with us performing some basic HOL
  instructions. Then I put the findings, code and output into Latex to
  further enhance my knowledge on making these reports. I did the
  exercises from the PDF book from the class 4-6-3, 4-6-4, 5.3.4,
  5.3.5, and 6.2.1. Below are the sections that are in this report for
  each problem (some sections have Execution Transcripts too:
  \begin{itemize}
  \item Problem statement
  \item Relevant code
  \item Test results
  \item Execution Transcripts
  \end{itemize}
  
  For each problem or exercise-oriented chapter in the main body of
  the report is a corresponding chapter in the Appendix containing the
  source code in ML.  This source code is not pasted into the
  Appendix.  Rather, it is input directly from the source code file
  itself.

\end{abstract}

\begin{acknowledgments}
  I would like to acknowledge the 2 professors, Professor Chin and 
  Professor Hamner, that have helped me begin to understand this new
  ML programming language. Also to Syracuse University for accepting
  me to this Masters program in Cybersecurity.
\end{acknowledgments}

\tableofcontents{}


\chapter{Executive Summary}
\label{cha:executive-summary}

\textbf{All requirements for this project are satisfied}.
Specifically,
\begin{description}
\item[Report Contents] \ \\
  The report has the following content:
  \begin{enumerate}[{}]
  \item Chapter~\ref{cha:executive-summary}: Executive Summary
  \item Chapter~\ref{cha:exercise-4-6-3}: Exercise 4.6.3
    \begin{enumerate}[{}]
    \item Section~\ref{sec:problem-statement-ex-4-6-3}: Problem statement
    \item Section~\ref{sec:relevant-code-ex-4-6-3}: Relevant code
    \item Section~\ref{sec:tests-ex-4-6-3}: Test results
    \item Section~\ref{sec:exe-ex-4-6-3}: Execution Transcripts
    \end{enumerate}
  \item Chapter~\ref{cha:exercise-4-6-4}: Exercise 4.6.4
    \begin{enumerate}[{}]
    \item Section~\ref{sec:problem-statement-ex-4-6-4}: Problem statement
    \item Section~\ref{sec:relevant-code-ex-4-6-4}: Relevant code
    \item Section~\ref{sec:tests-ex-4-6-4}: Test results
    \item Section~\ref{sec:exe-ex-4-6-4}: Execution Transcripts
    \end{enumerate}
  \item Chapter~\ref{cha:exercise-5-3-4}: Exercise 5.3.4
    \begin{enumerate}[{}]
    \item Section~\ref{sec:problem-statement-ex-5-3-4}: Problem statement
    \item Section~\ref{sec:relevant-code-ex-5-3-4}: Relevant code
    \item Section~\ref{sec:tests-ex-5-3-4}: Test results
    \item Section~\ref{sec:exe-ex-5-3-4}: Execution Transcripts
    \end{enumerate}
  \item Chapter~\ref{cha:exercise-5-3-5}: Exercise 5.3.5
    \begin{enumerate}[{}]
    \item Section~\ref{sec:problem-statement-ex-5-3-5}: Problem statement
    \item Section~\ref{sec:relevant-code-ex-5-3-5}: Relevant code
    \item Section~\ref{sec:tests-ex-5-3-5}: Test results
    \item Section~\ref{sec:exe-ex-5-3-5}: Execution Transcripts
    \end{enumerate}
  \item Chapter~\ref{cha:exercise-6-2-1}: Exercise 6.2.1
    \begin{enumerate}[{}]
    \item Section~\ref{sec:problem-statement-ex-6-2-1}: Problem statement
    \item Section~\ref{sec:relevant-code-ex-6-2-1}: Relevant code
    \item Section~\ref{sec:tests-ex-6-2-1}: Test results
    \end{enumerate}
  \item Chapter~\ref{cha:source-code-ex-4-6-3}: Source Code for Exercise
    4.6.3
  \item Chapter~\ref{cha:source-code-ex-4-6-4}: Source Code for Exercise
    4.6.4
  \item Chapter~\ref{cha:source-code-ex-5-3-4}: Source Code for Exercise
    5.3.4
  \item Chapter~\ref{cha:source-code-ex-5-3-5}: Source Code for Exercise
    5.3.5
  \item Chapter~\ref{cha:source-code-ex-6-2-1}: Source Code for Exercise
    6.2.1
  \end{enumerate}
\item[Reproducibility in ML and \LaTeX{}] \ \\
  Our ML and \LaTeX{} source files compile with no errors.
\end{description}


\chapter{Exercise 4.6.3}
\label{cha:exercise-4-6-3}

\section{Problem Statement for Exercise 4.6.3}
\label{sec:problem-statement-ex-4-6-3}
For this problem I solved the Exercise 4.6.3 from the PDF book.  This
problem was to create fucntions for the 5 probelms from Exercise
4.6.2. I had to create the functions using val and then using fun.

\section{Relevant Code for Exercise 4.6.3}
\label{sec:relevant-code-ex-4-6-3}
  The relevant code wil be in the corresponding Appendix at the
  end of the report.

\section{Test cases for Exercise 4.6.3}
\label{sec:tests-ex-4-6-3}

Below are the test cases for this exercise that were requested based
of the provided handout. I did change variable names where needed.

Part A: Test List A = [(1,2,3), (4,5,6), (7,8,9)] 
Part B: Test List B = [(0,0), (1,2), (4,3)]
Part C: Test List C = [(“Hi”, “ there!”), (“Oh “, “no!”), (“What”, “ the …”)]
Part D: Test List D1 = [([0,1], [2,3,4]), ([], [0,1])]
Part D: Test List D2 = [([true, true], [])]
Part E: Test List E = [(2,1), (5,5), (5,10)]

\section{Execution Transcripts for Exercise 4.6.3}
\label{sec:exe-ex-4-6-3}

Below are the results from running the test cases:

The following is output from \emph{ex-4-6-3.sml}
\lstinputlisting{ML/Ex-4-6-3.trans}

\chapter{Exercise 4.6.4}
\label{cha:exercise-4-6-4}

\section{Problem Statement for Exercise 4.6.4}
\label{sec:problem-statement-ex-4-6-4}
For this problem I solved the Exercise 4.6.4 from the PDF book.  This
problem was to creat a funtion called listSquares that would take in a
list of values and return the square of each value in the list.

\section{Relevant Code for Exercise 4.6.4}
\label{sec:relevant-code-ex-4-6-4}
  The relevant code wil be in the corresponding Appendix at the
  end of the report.

\section{Test cases for Exercise 4.6.4}
\label{sec:tests-ex-4-6-4}
Below are the test case that was used and provided by the professor.

Test List = [1,2,3,4,5]

\section{Execution Transcripts for Exercise 4.6.4}
\label{sec:exe-ex-4-6-4}

Below are the results from running the test cases:

The following is output from \emph{ex-4-6-4.sml}
\lstinputlisting{ML/Ex-4-6-4.trans}

\chapter{Exercise 5.3.4}
\label{cha:exercise-5-3-4}

\section{Problem Statement for Exercise 5.3.4}
\label{sec:problem-statement-ex-5-3-4}
For this problem I solved the Exercise 5.3.4 from the PDF book.  This
problem was to set up a filter using the rules from the previous
Exercise 5.3.3.

\section{Relevant Code for Exercise 5.3.4}
\label{sec:relevant-code-ex-5-3-4}
  The relevant code wil be in the corresponding Appendix at the
  end of the report.

\section{Test cases for Exercise 5.3.4}
\label{sec:tests-ex-5-3-4}
Below are the test case that was used and provided by the professor.

testResults = Filter(fn x => x <5) [1,2,3,4,5,6,7,8,9]

\section{Execution Transcripts for Exercise 5.3.4}
\label{sec:exe-ex-5-3-4}

Below are the results from running the test cases:

The following is output from \emph{ex-5-3-4.sml}
\lstinputlisting{ML/Ex-5-3-4.trans}

\chapter{Exercise 5.3.5}
\label{cha:exercise-5-3-5}

\section{Problem Statement for Exercise 5.3.5}
\label{sec:problem-statement-ex-5-3-5}
For this problem I solved the Exercise 5.3.5 from the PDF book. The
problem was to take in a list of integers and an integer and then
based off that integer return the list with only values that are
greater than the solo integer.

\section{Relevant Code for Exercise 5.3.5}
\label{sec:relevant-code-ex-5-3-5}
  The relevant code wil be in the corresponding Appendix at the
  end of the report.

\section{Test cases for Exercise 5.3.5}
\label{sec:tests-ex-5-3-5}
Below are the test case that was used and provided by the professor.

addPairsGreaterThan 0 [(0,1), (2,0), (2,3), (4,5)];

\section{Execution Transcripts for Exercise 5.3.5}
\label{sec:exe-ex-5-3-5}
Below are the results from running the test cases:

The following is output from \emph{ex-5-3-5.sml}
\lstinputlisting{ML/Ex-5-3-5.trans}

\chapter{Exercise 6.2.1}
\label{cha:exercise-6-2-1}

\section{Problem Statement for Exercise 6.2.1}
\label{sec:problem-statement-ex-6-2-1}
For this problem I solved the Exercise 6.2.1 from the PDF book. This
problem was to solve 7 different HOL based problems. I made sure to
show types and disable Unicode so the printing of the output was as
expected/wanted.

\section{Relevant Code for Exercise 6.2.1}
\label{sec:relevant-code-ex-6-2-1}
  The relevant code wil be in the corresponding Appendix at the
  end of the report.

\section{Test cases for Exercise 6.2.1}
\label{sec:tests-ex-6-2-1}
In this section we did not necessarily run any tests we just got the
output from running the HOL commands.

\chapter{Relevant Information}
\label{cha:relevant-information}

\section{Specific Test Cases}
\label{sec:specific-test-cases}

\subsection{Test Cases for Exercise 4.6.3}
\label{sec:test-cases-ex-4-6-3}
Below are the test cases for this exercise that were requested based
of the provided handout. I did change variable names where needed.

Part A: Test List A = [(1,2,3), (4,5,6), (7,8,9)] 
Part B: Test List B = [(0,0), (1,2), (4,3)]
Part C: Test List C = [(“Hi”, “ there!”), (“Oh “, “no!”), (“What”, “ the …”)]
Part D: Test List D1 = [([0,1], [2,3,4]), ([], [0,1])]
Part D: Test List D2 = [([true, true], [])]
Part E: Test List E = [(2,1), (5,5), (5,10)]

\subsection{Test Cases for Exercise 4.6.4}
\label{sec:test-cases-ex-4-6-4}
Below are the test case that was used and provided by the professor.

Test List = [1,2,3,4,5]

\subsection{Test Cases for Exercise 5.3.4}
\label{sec:test-cases-ex-5-3-4}
Below are the test case that was used and provided by the professor.

testResults = Filter(fn x => x <5) [1,2,3,4,5,6,7,8,9]

\subsection{Test Cases for Exercise 5.3.5}
\label{sec:test-cases-ex-5-3-5}
Below are the test case that was used and provided by the professor.

addPairsGreaterThan 0 [(0,1), (2,0), (2,3), (4,5)];

\subsection{Test Cases for Exercise 6.2.1}
\label{sec:test-cases-ex-6-2-1}
In this section we did not necessarily run any tests we just got the
output from running the HOL commands.


%% ------------------------------------------
%% this restarts the section numbering and
%% changes chapter numbering to letters starting
%% with A
%% ------------------------------------------
\appendix{} 


\chapter{Source Code for  Exercise 4.6.3}
\label{cha:source-code-ex-4-6-3}

The following code is from \emph{Ex-4-6-3.sml}
\lstinputlisting{ML/Ex-4-6-3.sml}

\chapter{Source Code for  Exercise 4.6.4}
\label{cha:source-code-ex-4-6-4}

The following code is from \emph{Ex-4-6-4.sml}
\lstinputlisting{ML/Ex-4-6-4.sml}

\chapter{Source Code for  Exercise 5.3.4}
\label{cha:source-code-ex-5-3-4}

The following code is from \emph{Ex-5-3-4.sml}
\lstinputlisting{ML/Ex-5-3-4.sml}

\chapter{Source Code for  Exercise 5.3.5}
\label{cha:source-code-ex-5-3-5}

The following code is from \emph{Ex-5-3-5.sml}
\lstinputlisting{ML/Ex-5-3-5.sml}

\chapter{Source Code for  Exercise 6.2.1}
\label{cha:source-code-ex-6-2-1}

The following code is from \emph{Ex-6-2-1.sml}
\lstinputlisting{ML/Ex-6-2-1.sml}

\end{document}