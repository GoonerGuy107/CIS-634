%% ---------------------------------------------------
%% Kyle Peppe
%% CIS 634 Project 6 Report
%% ---------------------------------------------------
\documentclass{report}

\title{Project 6 Report}
\author{Kyle Peppe}
\date{\today}

%% ---------------------------------------------------
%% 634format specifies the format of our reports
%% ---------------------------------------------------
\usepackage{634format}

%% ---------------------------------------------------
%% enumerate 
%% ---------------------------------------------------
\usepackage{enumerate}

%% ---------------------------------------------------
%% listings is used for including our source code in reports
%% textcomp provides additional symbols
%% ---------------------------------------------------
\usepackage{listings}
\usepackage{textcomp}

%% ---------------------------------------------------
%% Packages for math environments
%% ---------------------------------------------------
\usepackage{amsmath}

%% ---------------------------------------------------
%% Packages for URLs and hotlinks in the table of contents
%% and symbolic cross references using \ref
%% ---------------------------------------------------
\usepackage{hyperref}

%% ---------------------------------------------------
%% Packages for using HOL-generated macros and displays
%% ---------------------------------------------------
\usepackage{holtex}
\usepackage{holtexbasic}
\input{commands}

\begin{document}

%% --------------------------------------------------- 
%% The listings  parameter "language" is set to "ML"
%% ---------------------------------------------------
\lstset{language=ML}


\maketitle{}

\begin{abstract}
  This project was to demonstrate my command of proving several
  different theories using the tacticals, types, and in Chapter 13
  doing slightly simpler goal oriented proofs. I have completed the
  exercises 13.10.1, 13.10.2, and 14.4.1.
        This project includes the following packages:
	\begin{description}
		\item[\emph{634format.sty}] A format style for this course
		\item[\emph{listings}] Package for displaying and inputting ML source code
		\item[\emph{holtex}] HOL style files and commands to display in the report
	\end{description}
        This document also demonstrates my ability to :
	\begin{itemize}
		\item Easily generate a table of contents,
		\item Refer to chapter and section labels
	\end{itemize}
\end{abstract}

\tableofcontents{}

\begin{acknowledgments}
  I would like to acknowledge the 2 professors, Professor Chin and 
  Professor Hamner, that have helped me begin to understand this new
  ML programming language. Also to Syracuse University for accepting
  me to this Masters program in Cybersecurity.
\end{acknowledgments}

\chapter{Executive Summary}
\label{cha:executive-summary}
\textbf{All requirements for this project are satisfied.}
Specifically,
\begin{description}
\item[Report Contents] \ \\
  Our report has the following content:
  \begin{enumerate}[{}]
  \item Chapter~\ref{cha:executive-summary}: Executive Summary
  \item Chapter~\ref{cha:13-10-1}: Exercise 13.10.1
    \begin{enumerate}[{}]
    \item Section~\ref{problem-statement-13-10-1}: Problem Statement
    \item Section~\ref{forward-13-10-1-A}: Forward proof of theorem aclExercise1
    \item Section~\ref{prove-13-10-1-B}: USE PROVE TAC only to prove theorem aclExercise1B
    \item Section~\ref{goal-13-10-1-C}: Goal Oriented proof using ACL tactics
    \end{enumerate}
  \item Chapter~\ref{cha:13-10-2}: Exercise 13.10.2
    \begin{enumerate}[{}]
    \item Section~\ref{problem-statement-13-10-2}: Problem Statement
    \item Section~\ref{foward-13-10-2-A}: Forward proof of theorem aclExercise2
    \item Section~\ref{prove-13-10-2-B}: Use PROVE TAC only to prove theorem aclExercise2B
    \item Section~\ref{goal-13-10-2-C}: Goal Oriented proof using ACL tactics
    \end{enumerate}
  \item Chapter~\ref{cha:14-4-1}: Exercise 14.4.1
    \begin{enumerate}[{}]
    \item Section~\ref{problem-statement-14-4-1}: Problem Statement
    \item Section~\ref{def-14-4-1}: Definition of datatypes
    \item Section~\ref{proof-14-4-1-A}: Proof of OpRuleLaunch thm
    \item Section~\ref{proof-14-4-1-B}: Proof of OpRuleAbort thm
    \item Section~\ref{proof-14-4-1-C}: Proof of ApRuleActivate thm
    \item Section~\ref{proof-14-4-1-D}: Proof of ApRuleStandDown thm
    \end{enumerate}
 \item Appendix~\ref{cha:source-code-ex1}: Source Code for Example1Script
 \item Appendix~\ref{cha:source-code-13-10-1-2}: Source Code for 13.10.1 and 13.10.2
 \item Appendix~\ref{cha:source-code-14-4-1}: Source Code for 14.4.1
  \end{enumerate}
\item[Reproducibility in ML and \LaTeX{}] \ \\
  The ML and \LaTeX{} source files compile with no errors.
\end{description}


\chapter{Excercise 13.10.1}
\label{cha:13-10-1}

\section{Problem statement}
\label{problem-statement-13-10-1}
Do a goal-oriented proof for parts A, B, and C: 

Alice says go  Bob says go 
Alice and Bob says go 

For each part we use first the inference
rules, then for part B just prove_tac and for part C using specific
ACL tactics

\section{Forward proof of theorem aclExercise1}
\label{forward-13-10-1-A}
\begin{lstlisting}[frame=TBlr]
(* Exercise 1		*)
val aclExercise1 =
let
	val th1 = ACL_ASSUM``((Name Alice) says (prop go)):
	(commands,staff,'d,'e)Form``
 	val th2 = ACL_ASSUM``((Name Bob) says (prop go)):
	(commands,staff,'d,'e)Form``
 	val th3 = ACL_CONJ th1 th2
 	val th4 = AND_SAYS_RL th3
 	val th5 = DISCH(hd(hyp th2)) th4
in
	DISCH(hd(hyp th1)) th5
end;
\end{lstlisting}

\begin{session}
  \begin{scriptsize}
\begin{verbatim}
Solution/Theorem
# # # # # # # # # # # val aclExercise1 =
   |- ((M :(commands, 'b, staff, 'd, 'e) Kripke),(Oi :'d po),
    (Os :'e po)) sat
   Name Alice says (prop go :(commands, staff, 'd, 'e) Form) ==>
   (M,Oi,Os) sat
   Name Bob says (prop go :(commands, staff, 'd, 'e) Form) ==>
   (M,Oi,Os) sat
   Name Alice meet Name Bob says
   (prop go :(commands, staff, 'd, 'e) Form):
   thm
\end{verbatim}
  \end{scriptsize}
\end{session}

\section{USE PROVE TAC only to prove theorem aclExercise1B}
\label{prove-13-10-1-B}
\begin{lstlisting}[frame=TBlr]
(* Exercise 1A		*)
val aclExercise1A =
TAC_PROOF(([],
	``((M :(commands, 'b, staff, 'd, 'e) Kripke),(Oi :'d po),
	  (Os :'e po)) sat Name Alice says (prop go) ==>
  	  (M,Oi,Os) sat Name Bob says (prop go) ==>
  	  (M,Oi,Os) sat (Name Alice) meet (Name Bob) says (prop go)``),
	  PROVE_TAC[Conjunction, And_Says_Eq]
	 )
\end{lstlisting}

\begin{session}
  \begin{scriptsize}
\begin{verbatim}
Solution/Theorem
# # # # # # # Meson search level: .....
val aclExercise1A =
   |- ((M :(commands, 'b, staff, 'd, 'e) Kripke),(Oi :'d po),
    (Os :'e po)) sat
   Name Alice says (prop go :(commands, staff, 'd, 'e) Form) ==>
   (M,Oi,Os) sat
   Name Bob says (prop go :(commands, staff, 'd, 'e) Form) ==>
   (M,Oi,Os) sat
   Name Alice meet Name Bob says
   (prop go :(commands, staff, 'd, 'e) Form):
   thm
\end{verbatim}
  \end{scriptsize}
\end{session}

\section{Goal Oriented proof using ACL tactics}
\label{goal-13-10-1-C}
\begin{lstlisting}[frame=TBlr]
(* Exercise 1B		*)
val aclExercise1B = 
TAC_PROOF(([],
	``((M :(commands, 'b, staff, 'd, 'e) Kripke),(Oi :'d po),
	  (Os :'e po)) sat Name Alice says (prop go) ==>
  	  (M,Oi,Os) sat Name Bob says (prop go) ==>
  	  (M,Oi,Os) sat (Name Alice) meet (Name Bob) says (prop go)``),
	  REPEAT STRIP_TAC THEN
	  ACL_AND_SAYS_RL_TAC THEN
	  ACL_CONJ_TAC THEN
	  PROVE_TAC[]
	 )
\end{lstlisting}

\begin{session}
  \begin{scriptsize}
\begin{verbatim}
Solution/Theorem
# # # # # # # # # # Meson search level: ..
Meson search level: ..
val aclExercise1B =
   |- ((M :(commands, 'b, staff, 'd, 'e) Kripke),(Oi :'d po),
    (Os :'e po)) sat
   Name Alice says (prop go :(commands, staff, 'd, 'e) Form) ==>
   (M,Oi,Os) sat
   Name Bob says (prop go :(commands, staff, 'd, 'e) Form) ==>
   (M,Oi,Os) sat
   Name Alice meet Name Bob says
   (prop go :(commands, staff, 'd, 'e) Form):
   thm
\end{verbatim}
  \end{scriptsize}
\end{session}

\chapter{Excercise 13.10.2}
\label{cha:13-10-2}

\section{Problem statement}
\label{problem-statement-13-10-2}
Do a goal-oriented proof for parts A, B, and C: 

Alice says go   Alice Controls go  go implies launch 
Bob says launch 

For each part we use first the inference
rules, then for part B just prove_tac and for part C using specific
ACL tactics

\section{Forward proof of theorem aclExercise2}
\label{foward-13-10-2-A}
\begin{lstlisting}[frame=TBlr]
val aclExercise2 =
let
	val th1 = ACL_ASSUM``((Name Alice) says (prop go)):
	(commands,staff,'d,'e)Form``
 	val th2 = ACL_ASSUM``((Name Alice) controls (prop go)):
	(commands,staff,'d,'e)Form``
 	val th3 = ACL_ASSUM``((prop go) impf (prop launch)):
	(commands,staff,'d,'e)Form``
 	val th4 = CONTROLS th2 th1
 	val th5 = ACL_MP th4 th3
 	val th6 = SAYS ``(Name Bob):staff Princ`` th5
 	val th7 = DISCH(hd(hyp th3)) th6
 	val th8 = DISCH(hd(hyp th2)) th7
in
	DISCH(hd(hyp th1)) th8
end;
\end{lstlisting}

\begin{session}
  \begin{scriptsize}
\begin{verbatim}
# # # # # # # # # # # # # # # val aclExercise2 =
   |- ((M :(commands, 'b, staff, 'd, 'e) Kripke),(Oi :'d po),
    (Os :'e po)) sat
   Name Alice says (prop go :(commands, staff, 'd, 'e) Form) ==>
   (M,Oi,Os) sat
   Name Alice controls (prop go :(commands, staff, 'd, 'e) Form) ==>
   (M,Oi,Os) sat
   (prop go :(commands, staff, 'd, 'e) Form) impf
   (prop launch :(commands, staff, 'd, 'e) Form) ==>
   (M,Oi,Os) sat
   Name Bob says (prop launch :(commands, staff, 'd, 'e) Form):
   thm
\end{verbatim}
  \end{scriptsize}
\end{session}

\section{Use PROVE TAC only to prove theorem aclExercise2B}
\label{prove-13-10-2-B}
\begin{lstlisting}[frame=TBlr]
val aclExercise2A =
TAC_PROOF(([],
	``((M :(commands, 'b, staff, 'd, 'e) Kripke),(Oi :'d po),
	  (Os :'e po)) sat Name Alice says (prop go) ==>
  	  (M,Oi,Os) sat Name Alice controls (prop go) ==>
  	  (M,Oi,Os) sat (prop go) impf (prop launch) ==>
  	  (M,Oi,Os) sat Name Bob says (prop launch)``),
	  PROVE_TAC[Controls, Modus_Ponens, Says]
	  );
\end{lstlisting}

\begin{session}
  \begin{scriptsize}
\begin{verbatim}
# # # # # # # # Meson search level: .......
val aclExercise2A =
   |- ((M :(commands, 'b, staff, 'd, 'e) Kripke),(Oi :'d po),
    (Os :'e po)) sat
   Name Alice says (prop go :(commands, staff, 'd, 'e) Form) ==>
   (M,Oi,Os) sat
   Name Alice controls (prop go :(commands, staff, 'd, 'e) Form) ==>
   (M,Oi,Os) sat
   (prop go :(commands, staff, 'd, 'e) Form) impf
   (prop launch :(commands, staff, 'd, 'e) Form) ==>
   (M,Oi,Os) sat
   Name Bob says (prop launch :(commands, staff, 'd, 'e) Form):
   thm
\end{verbatim}
  \end{scriptsize}
\end{session}

\section{Goal Oriented proof using ACL tactics}
\label{goal-13-10-2-C}
\begin{lstlisting}[frame=TBlr]
val aclExercise2B =
TAC_PROOF(([],
	``((M :(commands, 'b, staff, 'd, 'e) Kripke),(Oi :'d po),
	  (Os :'e po)) sat Name Alice says (prop go) ==>
  	  (M,Oi,Os) sat Name Alice controls (prop go) ==>
  	  (M,Oi,Os) sat (prop go) impf (prop launch) ==>
  	  (M,Oi,Os) sat Name Bob says (prop launch)``),
	  REPEAT STRIP_TAC THEN
	  ACL_SAYS_TAC THEN
	  PAT_ASSUM ``(M,Oi,Os) sat Name Alice says (prop go)``
	  (fn th1 => (PAT_ASSUM ``(M,Oi,Os) sat Name Alice controls
	  (prop go)`` (fn th2 => ASSUME_TAC(CONTROLS th2 th1)))) THEN
	  PAT_ASSUM ``(M,Oi,Os) sat (prop go)``
	  (fn th1 => (PAT_ASSUM ``(M,Oi,Os) sat (prop go) impf (prop launch)``
	  (fn th2 => PROVE_TAC[(ACL_MP th1 th2)])))
	 )
\end{lstlisting}

\begin{session}
  \begin{scriptsize}
\begin{verbatim}
# # # # # # # # # # # # # # # <<HOL message: inventing new type variable names: 'a, 'b, 'c>>
<<HOL message: inventing new type variable names: 'a, 'b, 'c, 'd>>
Meson search level: ..
val aclExercise2B =
   |- ((M :(commands, 'b, staff, 'd, 'e) Kripke),(Oi :'d po),
    (Os :'e po)) sat
   Name Alice says (prop go :(commands, staff, 'd, 'e) Form) ==>
   (M,Oi,Os) sat
   Name Alice controls (prop go :(commands, staff, 'd, 'e) Form) ==>
   (M,Oi,Os) sat
   (prop go :(commands, staff, 'd, 'e) Form) impf
   (prop launch :(commands, staff, 'd, 'e) Form) ==>
   (M,Oi,Os) sat
   Name Bob says (prop launch :(commands, staff, 'd, 'e) Form):
   thm
\end{verbatim}
  \end{scriptsize}
\end{session}


\chapter{Excercise 14.4.1}
\label{cha:14-4-1}

\section{Problem statement}
\label{problem-statement-14-4-1}
Since the actual HOL Theorems are too long to individually lay out
here I am just giving the 4 theorem names that we were given to solve:
OpRuleLaunch_thm
OpRuleAbort_thm
ApRuleActivate_thm
ApRuleStandDown_thm

\section{Definition of datatypes}
\label{def-14-4-1}
\begin{lstlisting}[frame=TBlr]
val _ = Datatype `commands = go | nogo | launch | abort | activate | stand_down`
val _ = Datatype `people = Alice | Bob`
val _ = Datatype `roles = Commander | Operator | CA`
val _ = Datatype `keyPrinc = Staff people | Role roles | Ap num`
val _ = Datatype `principals = PR keyPrinc | Key keyPrinc`
\end{lstlisting}
\begin{session}
  \begin{scriptsize}
\begin{verbatim}
<<HOL message: Defined type: "commands">>
> <<HOL message: Defined type: "people">>
> <<HOL message: Defined type: "roles">>
> <<HOL message: Defined type: "keyPrinc">>
> <<HOL message: Defined type: "principals">>
\end{verbatim}
  \end{scriptsize}
\end{session}

\section{Proof of OpRuleLaunch thm}
\label{proof-14-4-1-A}
\begin{lstlisting}[frame=TBlr]
val OpRuleLaunch_thm =
let
	val th1 = ACL_ASSUM ``((Name (PR (Role Commander))) 
        controls (prop go)) : (commands,principals,'d,'e)Form``
	val th2 = ACL_ASSUM ``(reps(Name (PR (Staff Alice))) 
        (Name (PR (Role Commander))) (prop go)) 
        : (commands,principals,'d,'e)Form``
	val th3 = ACL_ASSUM ``((Name (Key (Staff Alice))) 
        quoting (Name (PR (Role Commander))) says (prop go)) 
        : (commands,principals,'d,'e)Form``
	val th4 = ACL_ASSUM ``((prop go) impf (prop launch)) 
        : (commands,principals,'d,'e)Form``
	val th5 = ACL_ASSUM ``((Name (Key (Role CA))) 
        speaks_for (Name (PR (Role CA)))) :
	(commands, principals,'d,'e)Form``
	val th6 = ACL_ASSUM ``((Name (Key (Role CA))) 
        says ((Name (Key (Staff Alice))) speaks_for 
        (Name (PR (Staff Alice))))) : (commands,principals,'d,'e)Form``
	val th7 = ACL_ASSUM ``((Name (PR (Role CA))) 
        controls ((Name (Key (Staff Alice))) speaks_for 
        (Name (PR (Staff Alice))))) : (commands,principals,'d,'e)Form``
	val th8 = SPEAKS_FOR th5 th6
	val th9 = CONTROLS th7 th8
	val th10 = IDEMP_SPEAKS_FOR ``Name (PR (Role Commander)) 
        : principals Princ``
	val th11 = INST_TYPE [``:'a`` |-> ``:commands``] th10
	val th12 = MONO_SPEAKS_FOR th9 th11
	val th13 = SPEAKS_FOR th12 th3
	val th14 = REPS th2 th13 th1
	val th15 = ACL_MP th14 th4
	val th16 = SAYS ``(Name (Key (Staff Bob))) quoting 
        (Name (PR (Role Operator))) : principals Princ`` th15
	val th17 = DISCH(hd(hyp th7)) th16
	val th18 = DISCH(hd(hyp th6)) th17
	val th19 = DISCH(hd(hyp th5)) th18
	val th20 = DISCH(hd(hyp th4)) th19
	val th21 = DISCH(hd(hyp th3)) th20
	val th22 = DISCH(hd(hyp th2)) th21
in
	DISCH(hd(hyp th1)) th22
end;
\end{lstlisting}

\begin{session}
  \begin{scriptsize}
\begin{verbatim}
# # # # # # # # # # # # # # # # # # # # # # # # # # # # # # # # # val OpRuleLaunch_thm =
   |- ((M :(commands, 'b, principals, 'd, 'e) Kripke),(Oi :'d po),
    (Os :'e po)) sat
   Name (PR (Role Commander)) controls
   (prop go :(commands, principals, 'd, 'e) Form) ==>
   (M,Oi,Os) sat
   reps (Name (PR (Staff Alice))) (Name (PR (Role Commander)))
     (prop go :(commands, principals, 'd, 'e) Form) ==>
   (M,Oi,Os) sat
   Name (Key (Staff Alice)) quoting Name (PR (Role Commander)) says
   (prop go :(commands, principals, 'd, 'e) Form) ==>
   (M,Oi,Os) sat
   (prop go :(commands, principals, 'd, 'e) Form) impf
   (prop launch :(commands, principals, 'd, 'e) Form) ==>
   (M,Oi,Os) sat
   ((Name (Key (Role CA)) speaks_for Name (PR (Role CA)))
      :(commands, principals, 'd, 'e) Form) ==>
   (M,Oi,Os) sat
   Name (Key (Role CA)) says
   ((Name (Key (Staff Alice)) speaks_for Name (PR (Staff Alice)))
      :(commands, principals, 'd, 'e) Form) ==>
   (M,Oi,Os) sat
   Name (PR (Role CA)) controls
   ((Name (Key (Staff Alice)) speaks_for Name (PR (Staff Alice)))
      :(commands, principals, 'd, 'e) Form) ==>
   (M,Oi,Os) sat
   Name (Key (Staff Bob)) quoting Name (PR (Role Operator)) says
   (prop launch :(commands, principals, 'd, 'e) Form):
   thm
\end{verbatim}
  \end{scriptsize}
\end{session}

\section{Proof of OpRuleAbort thm}
\label{proof-14-4-1-B}
\begin{lstlisting}[frame=TBlr]
val OpRuleAbort_thm =
let
	val th1 = ACL_ASSUM ``((Name (PR (Role Commander))) 
        controls (prop nogo)) : (commands,principals,'d,'e)Form``
	val th2 = ACL_ASSUM ``(reps(Name (PR (Staff Alice))) 
        (Name (PR (Role Commander))) (prop nogo)) 
        : (commands,principals,'d,'e)Form``
	val th3 = ACL_ASSUM ``((Name (Key (Staff Alice)))
        quoting (Name (PR (Role Commander))) says (prop nogo)) 
        : (commands,principals,'d,'e)Form``
	val th4 = ACL_ASSUM ``((prop nogo) impf (prop abort)) 
        : (commands,principals,'d,'e)Form``
	val th5 = ACL_ASSUM ``((Name (Key (Role CA))) 
        speaks_for (Name (PR (Role CA)))) : 
        (commands,principals,'d,'e)Form``
	val th6 = ACL_ASSUM ``((Name (Key (Role CA))) 
        says ((Name (Key (Staff Alice))) speaks_for 
        (Name (PR (Staff Alice))))) : (commands,principals,'d,'e)Form``
	val th7 = ACL_ASSUM ``((Name (PR (Role CA))) 
        controls ((Name (Key (Staff Alice))) speaks_for 
        (Name (PR (Staff Alice))))) : (commands,principals,'d,'e)Form``
 	val th8 = SPEAKS_FOR th5 th6
 	val th9 = CONTROLS th7 th8
 	val th10 = IDEMP_SPEAKS_FOR ``Name (PR (Role Commander)) 
        : principals Princ``
 	val th11 = INST_TYPE [``:'a`` |-> ``:commands``] th10
 	val th12 = MONO_SPEAKS_FOR th9 th11
 	val th13 = SPEAKS_FOR th12 th3
 	val th14 = REPS th2 th13 th1
 	val th15 = ACL_MP th14 th4
 	val th16 = SAYS ``(Name (Key (Staff Bob))) quoting 
        (Name (PR (Role Operator))) : principals Princ`` th15
 	val th17 = DISCH(hd(hyp th7)) th16
 	val th18 = DISCH(hd(hyp th6)) th17
 	val th19 = DISCH(hd(hyp th5)) th18
 	val th20 = DISCH(hd(hyp th4)) th19
 	val th21 = DISCH(hd(hyp th3)) th20
 	val th22 = DISCH(hd(hyp th2)) th21
in
	DISCH(hd(hyp th1)) th22
end;
\end{lstlisting}
\begin{session}
  \begin{scriptsize}
\begin{verbatim}
# # # # # # # # # # # # # # # # # # # # # # # # # # # # # # # # # val OpRuleAbort_thm =
   |- ((M :(commands, 'b, principals, 'd, 'e) Kripke),(Oi :'d po),
    (Os :'e po)) sat
   Name (PR (Role Commander)) controls
   (prop nogo :(commands, principals, 'd, 'e) Form) ==>
   (M,Oi,Os) sat
   reps (Name (PR (Staff Alice))) (Name (PR (Role Commander)))
     (prop nogo :(commands, principals, 'd, 'e) Form) ==>
   (M,Oi,Os) sat
   Name (Key (Staff Alice)) quoting Name (PR (Role Commander)) says
   (prop nogo :(commands, principals, 'd, 'e) Form) ==>
   (M,Oi,Os) sat
   (prop nogo :(commands, principals, 'd, 'e) Form) impf
   (prop abort :(commands, principals, 'd, 'e) Form) ==>
   (M,Oi,Os) sat
   ((Name (Key (Role CA)) speaks_for Name (PR (Role CA)))
      :(commands, principals, 'd, 'e) Form) ==>
   (M,Oi,Os) sat
   Name (Key (Role CA)) says
   ((Name (Key (Staff Alice)) speaks_for Name (PR (Staff Alice)))
      :(commands, principals, 'd, 'e) Form) ==>
   (M,Oi,Os) sat
   Name (PR (Role CA)) controls
   ((Name (Key (Staff Alice)) speaks_for Name (PR (Staff Alice)))
      :(commands, principals, 'd, 'e) Form) ==>
   (M,Oi,Os) sat
   Name (Key (Staff Bob)) quoting Name (PR (Role Operator)) says
   (prop abort :(commands, principals, 'd, 'e) Form):
   thm
\end{verbatim}
  \end{scriptsize}
\end{session}

\section{Proof of ApRuleActivate thm}
\label{proof-14-4-1-C}
\begin{lstlisting}[frame=TBlr]
val ApRuleActivate_thm =
let
	val th1 = ACL_ASSUM ``((Name (PR (Role Operator))) 
        controls (prop launch)) : (commands,principals,'d,'e)Form``
	val th2 = ACL_ASSUM ``(reps(Name (PR (Staff Bob))) 
        (Name (PR (Role Operator))) (prop launch)) : 
        (commands,principals,'d,'e)Form``
 	val th3 = ACL_ASSUM ``((Name (Key (Staff Bob))) 
        quoting (Name (PR (Role Operator))) says (prop launch)) 
        : (commands,principals,'d,'e)Form``
 	val th4 = ACL_ASSUM ``((prop launch) impf (prop activate)) 
        : (commands,principals,'d,'e)Form``
	val th5 = ACL_ASSUM ``((Name (Key (Role CA))) speaks_for 
        (Name (PR (Role CA)))) : (commands,principals,'d,'e)Form``
	val th6 = ACL_ASSUM ``((Name (Key (Role CA))) says 
        ((Name (Key (Staff Bob))) speaks_for (Name (PR (Staff Bob))))) 
        : (commands,principals,'d,'e)Form``
	val th7 = ACL_ASSUM ``((Name (PR (Role CA))) controls 
        ((Name (Key (Staff Bob)))  speaks_for (Name (PR (Staff Bob))))) 
        : (commands,principals,'d,'e)Form``
	val th8 = SPEAKS_FOR th5 th6
 	val th9 = CONTROLS th7 th8
 	val th10 = IDEMP_SPEAKS_FOR ``Name (PR (Role Operator)) 
        : principals Princ``
 	val th11 = INST_TYPE [``:'a`` |-> `` :commands``] th10
 	val th12 = MONO_SPEAKS_FOR th9 th11
 	val th13 = SPEAKS_FOR th12 th3
 	val th14 = REPS th2 th13 th1
 	val th15 = ACL_MP th14 th4
 	val th16 = DISCH(hd(hyp th7)) th15
 	val th17 = DISCH(hd(hyp th6)) th16
 	val th18 = DISCH(hd(hyp th5)) th17
 	val th19 = DISCH(hd(hyp th4)) th18
 	val th20 = DISCH(hd(hyp th3)) th19
 	val th21 = DISCH(hd(hyp th2)) th20
in
	DISCH(hd(hyp th1)) th21
end;
\end{lstlisting}
\begin{session}
  \begin{scriptsize}
\begin{verbatim}
# # # # # # # # # # # # # # # # # # # # # # # # # # # # # # # val ApRuleActivate_thm =
   |- ((M :(commands, 'b, principals, 'd, 'e) Kripke),(Oi :'d po),
    (Os :'e po)) sat
   Name (PR (Role Operator)) controls
   (prop launch :(commands, principals, 'd, 'e) Form) ==>
   (M,Oi,Os) sat
   reps (Name (PR (Staff Bob))) (Name (PR (Role Operator)))
     (prop launch :(commands, principals, 'd, 'e) Form) ==>
   (M,Oi,Os) sat
   Name (Key (Staff Bob)) quoting Name (PR (Role Operator)) says
   (prop launch :(commands, principals, 'd, 'e) Form) ==>
   (M,Oi,Os) sat
   (prop launch :(commands, principals, 'd, 'e) Form) impf
   (prop activate :(commands, principals, 'd, 'e) Form) ==>
   (M,Oi,Os) sat
   ((Name (Key (Role CA)) speaks_for Name (PR (Role CA)))
      :(commands, principals, 'd, 'e) Form) ==>
   (M,Oi,Os) sat
   Name (Key (Role CA)) says
   ((Name (Key (Staff Bob)) speaks_for Name (PR (Staff Bob)))
      :(commands, principals, 'd, 'e) Form) ==>
   (M,Oi,Os) sat
   Name (PR (Role CA)) controls
   ((Name (Key (Staff Bob)) speaks_for Name (PR (Staff Bob)))
      :(commands, principals, 'd, 'e) Form) ==>
   (M,Oi,Os) sat (prop activate :(commands, principals, 'd, 'e) Form):
   thm
\end{verbatim}
  \end{scriptsize}
\end{session}

\section{Proof of ApRuleStandDown thm}
\label{proof-14-4-1-D}
\begin{lstlisting}[frame=TBlr]
val ApRuleStandDown_thm =
let
	val th1 = ACL_ASSUM ``((Name (PR (Role Operator))) 
        controls (prop abort)) : (commands,principals,'d,'e)Form``
 	val th2 = ACL_ASSUM ``(reps(Name (PR (Staff Bob))) 
        (Name (PR (Role Operator))) (prop abort)) : 
        (commands,principals,'d,'e)Form``
 	val th3 = ACL_ASSUM ``((Name (Key (Staff Bob))) 
        quoting (Name (PR (Role Operator))) says (prop abort)) 
        : (commands,principals,'d,'e)Form``
 	val th4 = ACL_ASSUM ``((prop abort) impf (prop stand_down)) 
        : (commands,principals,'d,'e)Form``
 	val th5 = ACL_ASSUM ``((Name (Key (Role CA))) speaks_for 
        (Name (PR (Role CA)))) : (commands,principals,'d,'e)Form``
 	val th6 = ACL_ASSUM ``((Name (Key (Role CA))) says 
        ((Name (Key (Staff Bob))) speaks_for (Name (PR (Staff Bob))))) 
        : (commands,principals,'d,'e)Form``
 	val th7 = ACL_ASSUM ``((Name (PR (Role CA))) controls 
        ((Name (Key (Staff Bob))) speaks_for (Name (PR (Staff Bob))))) 
        : (commands,principals,'d,'e)Form``
 	val th8 = SPEAKS_FOR th5 th6
 	val th9 = CONTROLS th7 th8
 	val th10 = IDEMP_SPEAKS_FOR ``Name (PR (Role Operator)): 
        principals Princ``
 	val th11 = INST_TYPE [``:'a`` |-> ``:commands``] th10
 	val th12 = MONO_SPEAKS_FOR th9 th11
 	val th13 = SPEAKS_FOR th12 th3
 	val th14 = REPS th2 th13 th1
 	val th15 = ACL_MP th14 th4
 	val th16 = DISCH(hd(hyp th7)) th15
 	val th17 = DISCH(hd(hyp th6)) th16
 	val th18 = DISCH(hd(hyp th5)) th17
     	val th19 = DISCH(hd(hyp th4)) th18
     	val th20 = DISCH(hd(hyp th3)) th19
     	val th21 = DISCH(hd(hyp th2)) th20
in
	DISCH(hd(hyp th1)) th21
end;
\end{lstlisting}
\begin{session}
  \begin{scriptsize}
\begin{verbatim}
# # # # # # # # # # # # # # # # # # # # # # # # # # # # # # # val ApRuleStandDown_thm =
   |- ((M :(commands, 'b, principals, 'd, 'e) Kripke),(Oi :'d po),
    (Os :'e po)) sat
   Name (PR (Role Operator)) controls
   (prop abort :(commands, principals, 'd, 'e) Form) ==>
   (M,Oi,Os) sat
   reps (Name (PR (Staff Bob))) (Name (PR (Role Operator)))
     (prop abort :(commands, principals, 'd, 'e) Form) ==>
   (M,Oi,Os) sat
   Name (Key (Staff Bob)) quoting Name (PR (Role Operator)) says
   (prop abort :(commands, principals, 'd, 'e) Form) ==>
   (M,Oi,Os) sat
   (prop abort :(commands, principals, 'd, 'e) Form) impf
   (prop stand_down :(commands, principals, 'd, 'e) Form) ==>
   (M,Oi,Os) sat
   ((Name (Key (Role CA)) speaks_for Name (PR (Role CA)))
      :(commands, principals, 'd, 'e) Form) ==>
   (M,Oi,Os) sat
   Name (Key (Role CA)) says
   ((Name (Key (Staff Bob)) speaks_for Name (PR (Staff Bob)))
      :(commands, principals, 'd, 'e) Form) ==>
   (M,Oi,Os) sat
   Name (PR (Role CA)) controls
   ((Name (Key (Staff Bob)) speaks_for Name (PR (Staff Bob)))
      :(commands, principals, 'd, 'e) Form) ==>
   (M,Oi,Os) sat (prop stand_down :(commands, principals, 'd, 'e) Form):
   thm
\end{verbatim}
  \end{scriptsize}
\end{session}

\appendix{}

\chapter{Source code for Example1Script}
\label{cha:source-code-ex1}
\lstinputlisting{../HOL/example1Script.sml}
\chapter{Source code for 13.10.1 and 13.10.2}
\label{cha:source-code-13-10-1-2}
\lstinputlisting{../HOL/solutions1Script.sml}

\chapter{Source code for 14.4.1}
\label{cha:source-code-14-4-1}
\lstinputlisting{../HOL/conops0SolutionScript.sml}


\end{document}