%% ---------------------------------------------------
%% Kyle Peppe
%% CIS 634 Project 3 Report
%% Exercises 7.31, 7.3.2, and 7.3.3 only
%% ---------------------------------------------------
\documentclass{report}

\title{Project 3 Report}
\author{Kyle Peppe}
\date{\today}

%% ---------------------------------------------------
%% 634format specifies the format of our reports
%% ---------------------------------------------------
\usepackage{634format}

%% ---------------------------------------------------
%% enumerate 
%% ---------------------------------------------------
\usepackage{enumerate}

%% ---------------------------------------------------
%% listings is used for including our source code in reports
%% textcomp provides additional symbols
%% ---------------------------------------------------
\usepackage{listings}
\usepackage{textcomp}

%% ---------------------------------------------------
%% Packages for math environments
%% ---------------------------------------------------
\usepackage{amsmath}

%% ---------------------------------------------------
%% Packages for URLs and hotlinks in the table of contents
%% and symbolic cross references using \ref
%% ---------------------------------------------------
\usepackage{hyperref}

%% ---------------------------------------------------
%% Packages for using HOL-generated macros and displays
%% ---------------------------------------------------
\usepackage{holtex}
\usepackage{holtexbasic}
\input{commands}

\begin{document}

%% --------------------------------------------------- 
%% The listings  parameter "language" is set to "ML"
%% ---------------------------------------------------
\lstset{language=ML}


\maketitle{}

\begin{abstract}
  This project began my work in the HOL environment and using HOL
  terms. For these exercises I was creating a function that would
  start with a certain form and then returns a different form.
  \begin{itemize}
  \item Problem statement
  \item Relevant code
  \item Test results
  \item Execution Transcripts
  \end{itemize}
  
  For each problem or exercise-oriented chapter in the main body of
  the report is a corresponding chapter in the Appendix containing the
  source code in ML.  This source code is not pasted into the
  Appendix.  Rather, it is input directly from the source code file
  itself.

\end{abstract}

\begin{acknowledgments}
  I would like to acknowledge the 2 professors, Professor Chin and 
  Professor Hamner, that have helped me begin to understand this new
  ML programming language. Also to Syracuse University for accepting
  me to this Masters program in Cybersecurity.
\end{acknowledgments}

\tableofcontents{}


\chapter{Executive Summary}
\label{cha:executive-summary}

\textbf{All requirements for this project are satisfied}.
Specifically,
\begin{description}
\item[Report Contents] \ \\
  The report has the following content:
  \begin{enumerate}[{}]
  \item Chapter~\ref{cha:executive-summary}: Executive Summary
  \item Chapter~\ref{cha:exercise-7-3-1}: Exercise 7.3.1
    \begin{enumerate}[{}]
    \item Section~\ref{sec:problem-statement-ex-7-3-1}: Problem statement
    \item Section~\ref{sec:relevant-code-ex-7-3-1}: Relevant code
    \item Section~\ref{sec:tests-ex-7-3-1}: Test results
    \item Section~\ref{sec:exe-ex-7-3-1}: Execution Transcripts
    \end{enumerate}
  \item Chapter~\ref{cha:exercise-7-3-2}: Exercise 7.3.2
    \begin{enumerate}[{}]
    \item Section~\ref{sec:problem-statement-ex-7-3-2}: Problem statement
    \item Section~\ref{sec:relevant-code-ex-7-3-2}: Relevant code
    \item Section~\ref{sec:tests-ex-7-3-2}: Test results
    \item Section~\ref{sec:exe-ex-7-3-2}: Execution Transcripts
    \end{enumerate}
  \item Chapter~\ref{cha:exercise-7-3-3}: Exercise 7.3.3
    \begin{enumerate}[{}]
    \item Section~\ref{sec:problem-statement-ex-7-3-3}: Problem statement
    \item Section~\ref{sec:relevant-code-ex-7-3-3}: Relevant code
    \item Section~\ref{sec:tests-ex-7-3-3}: Test results
    \item Section~\ref{sec:exe-ex-7-3-3}: Execution Transcripts
    \end{enumerate}
  \item Chapter~\ref{cha:source-code-ex-7-3}: Source Code for Exercise
    7.3.1, 7.3.2, and 7.3.3
  \end{enumerate}
\item[Reproducibility in ML and \LaTeX{}] \ \\
  Our ML and \LaTeX{} source files compile with no errors.
\end{description}


\chapter{Exercise 7.3.1}
\label{cha:exercise-7-3-1}

\section{Problem Statement for Exercise 7.3.1}
\label{sec:problem-statement-ex-7-3-1}
For this problem I solved the Exercise 7.3.1 from the PDF book. We
were given a starting function and had to use the terms provided and
get that term to return a desired term.

\section{Relevant Code for Exercise 7.3.1}
\label{sec:relevant-code-ex-7-3-1}
  The relevant code wil be in the corresponding Appendix at the
  end of the report.

\section{Test cases for Exercise 7.3.1}
\label{sec:tests-ex-7-3-1}
There were no test cases here were giving a starting function and then
had to use logic to get to a desired funtion. The starting function
was:
\begin{lstlisting}[frame=TBlr]
andImp2Imp (p /\ q) ==> r
\end{lstlisting}

\section{Execution Transcripts for Exercise 7.3.1}
\label{sec:exe-ex-7-3-1}
Below are the results from running the test cases:

The following is output from \emph{chapter7Answer.sml}
\lstinputlisting{ML/Ex-7-3-1.trans}

\chapter{Exercise 7.3.2}
\label{cha:exercise-7-3-2}

\section{Problem Statement for Exercise 7.3.2}
\label{sec:problem-statement-ex-7-3-2}
For this problem I solved the Exercise 7.3.2 from the PDF book. In
this exercise we were given a term and instructed to get this to
return a new term similiar to 7.3.1. But there was the added step of
once getting our original term impImpAnd to return andImp2Imp then I
needed reverse andImp2Imp back to impImpAnd.

\section{Relevant Code for Exercise 7.3.2}
\label{sec:relevant-code-ex-7-3-2}
  The relevant code wil be in the corresponding Appendix at the
  end of the report.

\section{Test cases for Exercise 7.3.2}
\label{sec:tests-ex-7-3-2}
Once again we weren't necessarily given test cases but a started value
and an ending value to get the term to (and then reverse back):
\begin{lstlisting}[frame=TBlr]
impImpAnd = p ==> q ==> r
impImpAnd (andImp2Imp = p /\ q ==> r)
andImp2Imp (impImpAnd p ==> q ==> r)
\end{lstlisting}

\section{Execution Transcripts for Exercise 7.3.2}
\label{sec:exe-ex-7-3-2}
Below are the results from running the test cases:

The following is output from \emph{chapter7Answer.sml}
\lstinputlisting{ML/Ex-7-3-2.trans}

\chapter{Exercise 7.3.3}
\label{cha:exercise-7-3-3}

\section{Problem Statement for Exercise 7.3.3}
\label{sec:problem-statement-ex-7-3-3}
For this problem I solved the Exercise 7.3.3 from the PDF book. Here
we were given a term and had to get the function to return a different
form. This is similiar to the previous 2 exercises, just more
difficult.

\section{Relevant Code for Exercise 7.3.3}
\label{sec:relevant-code-ex-7-3-3}
  The relevant code wil be in the corresponding Appendix at the
  end of the report.

\section{Test cases for Exercise 7.3.3}
\label{sec:tests-ex-7-3-3}
Once again we weren't necessarily given test cases but a started value
and an ending value to get the term to (and then reverse back):
\begin{lstlisting}[frame=TBlr]
notExists = ~?z.Q(z)
\end{lstlisting}

\section{Execution Transcripts for Exercise 7.3.3}
\label{sec:exe-ex-7-3-3}
Below are the results from running the test cases:

The following is output from \emph{chapter7Answer.sml}
\lstinputlisting{ML/Ex-7-3-3.trans}


\chapter{Relevant Information}
\label{cha:relevant-information}

\section{Specific Test Cases}
\label{sec:specific-test-cases}

\subsection{Test Cases for Exercise 7.3.1}
\label{sec:test-cases-ex-7-3-1}
There were no test cases here were giving a starting function and then
had to use logic to get to a desired funtion. The starting function
was:
\begin{lstlisting}[frame=TBlr]
andImp2Imp (p /\ q) ==> r
\end{lstlisting}

\subsection{Test Cases for Exercise 7.3.2}
\label{sec:test-cases-ex-7-3-2}
Once again we weren't necessarily given test cases but a started value
and an ending value to get the term to (and then reverse back):
\begin{lstlisting}[frame=TBlr]
impImpAnd = p ==> q ==> r
impImpAnd (andImp2Imp = p /\ q ==> r)
andImp2Imp (impImpAnd p ==> q ==> r)
\end{lstlisting}

\subsection{Test Cases for Exercise 7.3.3}
\label{sec:test-cases-ex-7-3-3}
Once again we weren't necessarily given test cases but a started value
and an ending value to get the term to (and then reverse back):
\begin{lstlisting}[frame=TBlr]
notExists = ~?z.Q(z)
\end{lstlisting}

%% ------------------------------------------
%% this restarts the section numbering and
%% changes chapter numbering to letters starting
%% with A
%% ------------------------------------------
\appendix{} 


\chapter{Source Code for  Exercise 7.3.1, 7.3.2, and 7.3.3}
\label{cha:source-code-ex-7-3}

The following code is from \emph{chapter7Answer.sml}
\lstinputlisting{ML/chapter7Answer.sml}


\end{document}