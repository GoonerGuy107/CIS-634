%% ---------------------------------------------------
%% Kyle Peppe
%% CIS 634 Project 1 Report
%% ---------------------------------------------------
\documentclass{report}

\title{Project 1 Report}
\author{Kyle Peppe}
\date{\today}

%% ---------------------------------------------------
%% 634format specifies the format of our reports
%% ---------------------------------------------------
\usepackage{634format}

%% ---------------------------------------------------
%% enumerate 
%% ---------------------------------------------------
\usepackage{enumerate}

%% ---------------------------------------------------
%% listings is used for including our source code in reports
%% textcomp provides additional symbols
%% ---------------------------------------------------
\usepackage{listings}
\usepackage{textcomp}

%% ---------------------------------------------------
%% Packages for math environments
%% ---------------------------------------------------
\usepackage{amsmath}

%% ---------------------------------------------------
%% Packages for URLs and hotlinks in the table of contents
%% and symbolic cross references using \ref
%% ---------------------------------------------------
\usepackage{hyperref}

%% ---------------------------------------------------
%% Packages for using HOL-generated macros and displays
%% ---------------------------------------------------
\usepackage{holtex}
\usepackage{holtexbasic}
% =====================================================================
%
% Macros for typesetting the HOL system manual
%
% =====================================================================

% ---------------------------------------------------------------------
% Abbreviations for words and phrases
% ---------------------------------------------------------------------

\newcommand\TUTORIAL{{\footnotesize\sl TUTORIAL}}
\newcommand\DESCRIPTION{{\footnotesize\sl DESCRIPTION}}
\newcommand\REFERENCE{{\footnotesize\sl REFERENCE}}
\newcommand\LOGIC{{\footnotesize\sl LOGIC}}
\newcommand\LIBRARIES{{\footnotesize\sl LIBRARIES}}
\usepackage{textcomp}

\newcommand{\bs}{\texttt{\char'134}} % backslash
\newcommand{\lb}{\texttt{\char'173}} % left brace
\newcommand{\rb}{\texttt{\char'175}} % right brace
\newcommand{\td}{\texttt{\char'176}} % tilde
\newcommand{\lt}{\texttt{\char'74}} % less than
\newcommand{\gt}{\texttt{\char'76}} % greater than
\newcommand{\dol}{\texttt{\char'44}} % dollar
\newcommand{\pipe}{\texttt{\char'174}}
\newcommand{\apost}{\texttt{\textquotesingle}}
% double back quotes ``
\newcommand{\dq}{\texttt{\char'140\char'140}}
%These macros were included by slind:

\newcommand{\holquote}[1]{\dq#1\dq}

\def\HOL{\textsc{Hol}}
\def\holn{\HOL}  % i.e. hol n(inety-eight), no digits in
                 % macro names is a bit of a pain; deciding to do away
                 % with hol98 nomenclature means that we just want to
                 % write HOL for hol98.
\def\holnversion{Kananaskis-11}
\def\holnsversion{Kananaskis~11} % version with space rather than hyphen
\def\LCF{{\small LCF}}
\def\LCFLSM{{\small LCF{\kern-.2em}{\normalsize\_}{\kern0.1em}LSM}}
\def\PPL{{\small PP}{\kern-.095em}$\lambda$}
\def\PPLAMBDA{{\small PPLAMBDA}}
\def\ML{{\small ML}}
\def\holmake{\texttt{Holmake}}

\newcommand\ie{\mbox{\textit{i{.}e{.}}}}
\newcommand\eg{\mbox{\textit{e{.}g{.}}}}
\newcommand\viz{\mbox{viz{.}}}
\newcommand\adhoc{\mbox{\it ad hoc}}
\newcommand\etal{{\it et al.\/}}
% NOTE: \etc produces wrong spacing if used between sentences, that is
% like here \etc End such sentences with non-macro etc.
\newcommand\etc{\mbox{\textit{etc{.}}}}

% ---------------------------------------------------------------------
% Simple abbreviations and macros for mathematical typesetting
% ---------------------------------------------------------------------

\newcommand\fun{{\to}}
\newcommand\prd{{\times}}

\newcommand\conj{\ \wedge\ }
\newcommand\disj{\ \vee\ }
\newcommand\imp{ \Rightarrow }
\newcommand\eqv{\ \equiv\ }
\newcommand\cond{\rightarrow}
\newcommand\vbar{\mid}
\newcommand\turn{\ \vdash\ } % FIXME: "\ " results in extra space
\newcommand\hilbert{\varepsilon}
\newcommand\eqdef{\ \equiv\ }

\newcommand\natnums{\mbox{${\sf N}\!\!\!\!{\sf N}$}}
\newcommand\bools{\mbox{${\sf T}\!\!\!\!{\sf T}$}}

\newcommand\p{$\prime$}
\newcommand\f{$\forall$\ }
\newcommand\e{$\exists$\ }

\newcommand\orr{$\vee$\ }
\newcommand\negg{$\neg$\ }

\newcommand\arrr{$\rightarrow$}
\newcommand\hex{$\sharp $}

\newcommand{\uquant}[1]{\forall #1.\ }
\newcommand{\equant}[1]{\exists #1.\ }
\newcommand{\hquant}[1]{\hilbert #1.\ }
\newcommand{\iquant}[1]{\exists ! #1.\ }
\newcommand{\lquant}[1]{\lambda #1.\ }

\newcommand{\leave}[1]{\\[#1]\noindent}
\newcommand\entails{\mbox{\rule{.3mm}{4mm}\rule[2mm]{.2in}{.3mm}}}

% ---------------------------------------------------------------------
% Font-changing commands
% ---------------------------------------------------------------------

\newcommand{\theory}[1]{\hbox{{\small\tt #1}}}
\newcommand{\theoryimp}[1]{\texttt{#1}}

\newcommand{\con}[1]{{\sf #1}}
\newcommand{\rul}[1]{{\tt #1}}
\newcommand{\ty}[1]{\textsl{#1}}

\newcommand{\ml}[1]{\mbox{{\def\_{\char'137}\texttt{#1}}}}
\newcommand{\holtxt}[1]{\ml{#1}}
\newcommand\ms{\tt}
\newcommand{\s}[1]{{\small #1}}

\newcommand{\pin}[1]{{\bf #1}}
% FIXME: for multichar symbols \mathit should be used.
\def\m#1{\mbox{\normalsize$#1$}}

% ---------------------------------------------------------------------
% Abbreviations for particular mathematical constants etc.
% ---------------------------------------------------------------------

\newcommand\T{\con{T}}
\newcommand\F{\con{F}}
\newcommand\OneOne{\con{One\_One}}
\newcommand\OntoSubset{\con{Onto\_Subset}}
\newcommand\Onto{\con{Onto}}
\newcommand\TyDef{\con{Type\_Definition}}
\newcommand\Inv{\con{Inv}}
\newcommand\com{\con{o}}
\newcommand\Id{\con{I}}
\newcommand\MkPair{\con{Mk\_Pair}}
\newcommand\IsPair{\con{Is\_Pair}}
\newcommand\Fst{\con{Fst}}
\newcommand\Snd{\con{Snd}}
\newcommand\Suc{\con{Suc}}
\newcommand\Nil{\con{Nil}}
\newcommand\Cons{\con{Cons}}
\newcommand\Hd{\con{Hd}}
\newcommand\Tl{\con{Tl}}
\newcommand\Null{\con{Null}}
\newcommand\ListPrimRec{\con{List\_Prim\_Rec}}


\newcommand\SimpRec{\con{Simp\_Rec}}
\newcommand\SimpRecRel{\con{Simp\_Rec\_Rel}}
\newcommand\SimpRecFun{\con{Simp\_Rec\_Fun}}
\newcommand\PrimRec{\con{Prim\_Rec}}
\newcommand\PrimRecRel{\con{Prim\_Rec\_Rel}}
\newcommand\PrimRecFun{\con{Prim\_Rec\_Fun}}

\newcommand\bool{\ty{bool}}
\newcommand\num{\ty{num}}
\newcommand\ind{\ty{ind}}
\newcommand\lst{\ty{list}}

% ---------------------------------------------------------------------
% \minipagewidth = \textwidth minus 1.02 em
% ---------------------------------------------------------------------

\newlength{\minipagewidth}
\setlength{\minipagewidth}{\textwidth}
\addtolength{\minipagewidth}{-1.02em}

% ---------------------------------------------------------------------
% Environment for the items on the title page of a case study
% ---------------------------------------------------------------------

\newenvironment{inset}[1]{\noindent{\large\bf #1}\begin{list}%
{}{\setlength{\leftmargin}{\parindent}%
\setlength{\topsep}{-.1in}}\item }{\end{list}\vskip .4in}

% ---------------------------------------------------------------------
% Macros for little HOL sessions displayed in boxes.
%
% Usage: (1) \setcounter{sessioncount}{1} resets the session counter
%
%        (2) \begin{session}\begin{verbatim}
%             .
%              < lines from hol session >
%             .
%            \end{verbatim}\end{session}
%
%            typesets the session in a numbered box.
% ---------------------------------------------------------------------

\newlength{\hsbw}
\setlength{\hsbw}{\textwidth}
\addtolength{\hsbw}{-\arrayrulewidth}
\addtolength{\hsbw}{-\tabcolsep}
\newcommand\HOLSpacing{13pt}

\newcounter{sessioncount}
\setcounter{sessioncount}{0}

\newenvironment{session}{\begin{flushleft}
 \refstepcounter{sessioncount}
 \begin{tabular}{@{}|c@{}|@{}}\hline
 \begin{minipage}[b]{\hsbw}
 \vspace*{-.5pt}
 \begin{flushright}
 \rule{0.01in}{.15in}\rule{0.3in}{0.01in}\hspace{-0.35in}
 \raisebox{0.04in}{\makebox[0.3in][c]{\footnotesize\sl \thesessioncount}}
 \end{flushright}
 \vspace*{-.55in}
 \begingroup\small\baselineskip\HOLSpacing}{\endgroup\end{minipage}\\ \hline
 \end{tabular}
 \end{flushleft}}

% ---------------------------------------------------------------------
% Macro for boxed ML functions, etc.
%
% Usage: (1) \begin{holboxed}\begin{verbatim}
%               .
%               < lines giving names and types of mk functions >
%               .
%            \end{verbatim}\end{holboxed}
%
%            typesets the given lines in a box.
%
%            Conventions: lines are left-aligned under the "g" of begin,
%            and used to highlight primary reference for the ml function(s)
%            that appear in the box.
% ---------------------------------------------------------------------

\newenvironment{holboxed}{\begin{flushleft}
  \begin{tabular}{@{}|c@{}|@{}}\hline
  \begin{minipage}[b]{\hsbw}
% \vspace*{-.55in}
  \vspace*{.06in}
  \begingroup\small\baselineskip\HOLSpacing}{\endgroup\end{minipage}\\ \hline
  \end{tabular}
  \end{flushleft}}

% ---------------------------------------------------------------------
% Macro for unboxed ML functions, etc.
%
% Usage: (1) \begin{hol}\begin{verbatim}
%               .
%               < lines giving names and types of mk functions >
%               .
%            \end{verbatim}\end{hol}
%
%            typesets the given lines exactly like {boxed}, except there's
%            no box.
%
%            Conventions: lines are left-aligned under the "g" of begin,
%            and used to display ML code in verbatim, left aligned.
% ---------------------------------------------------------------------

\newenvironment{hol}{\begin{flushleft}
 \begin{tabular}{c@{}@{}}
 \begin{minipage}[b]{\hsbw}
% \vspace*{-.55in}
 \vspace*{.06in}
 \begingroup\small\baselineskip\HOLSpacing}{\endgroup\end{minipage}\\
 \end{tabular}
 \end{flushleft}}

% ---------------------------------------------------------------------
% Emphatic brackets
% ---------------------------------------------------------------------

\newcommand\leb{\lbrack\!\lbrack}
\newcommand\reb{\rbrack\!\rbrack}


% ---------------------------------------------------------------------
% Quotations
% ---------------------------------------------------------------------


%These macros were included by ap; they are used in Chapters 9 and 10
%of the HOL DESCRIPTION

\newcommand{\inds}%standard infinite set
 {\mbox{\rm I}}

\newcommand{\ch}%standard choice function
 {\mbox{\rm ch}}

\newcommand{\den}[1]%denotational brackets
 {[\![#1]\!]}

\newcommand{\two}%standard 2-element set
 {\mbox{\rm 2}}


\begin{document}

%% --------------------------------------------------- 
%% The listings  parameter "language" is set to "ML"
%% ---------------------------------------------------
\lstset{language=ML}


\maketitle{}

\begin{abstract}
  This project was as an introduction to the ML programming language 
  and some of the more basic commands that entails. I also then put
  our findings, code and output into Latex to further enhance our
  knowledge. I did the exercises from the PDF book from the class
  2-5-1, 3-4-1, and 3-4-2. Below are the sections that are in this
  report for each problem:
  \begin{itemize}
  \item Problem statement
  \item Relevant code
  \item Test results 
  \end{itemize}
  
  For each problem or exercise-oriented chapter in the main body of
  the report is a corresponding chapter in the Appendix containing the
  source code in ML.  This source code is not pasted into the
  Appendix.  Rather, it is input directly from the source code file
  itself.

\end{abstract}

\begin{acknowledgments}
  I would like to acknowledge the 2 professors, Professor Chin and 
  Professor Hamner, that have helped me begin to understand this new
  ML programming language. Also to Syracuse University for accepting
  me to this Masters program in Cybersecurity.
\end{acknowledgments}

\tableofcontents{}


\chapter{Executive Summary}
\label{cha:executive-summary}

\textbf{All requirements for this project are satisfied}.
Specifically,
\begin{description}
\item[Report Contents] \ \\
  The report has the following content:
  \begin{enumerate}[{}]
  \item Chapter~\ref{cha:executive-summary}: Executive Summary
  \item Chapter~\ref{cha:exercise-2-5-1}: Exercise 2.5.1
    \begin{enumerate}[{}]
    \item Section~\ref{sec:problem-statement-ex-2-5-1}: Problem statement
    \item Section~\ref{sec:relevant-code-ex-2-5-1}: Relevant code
    \item Section~\ref{sec:tests-2-5-1}: Test results
    \end{enumerate}
  \item Chapter~\ref{cha:exercise-3-4-1}: Exercise 3.4.1
    \begin{enumerate}[{}]
    \item Section~\ref{sec:problem-statement-ex-3-4-1}: Problem statement
    \item Section~\ref{sec:relevant-code-ex-3-4-1}: Relevant code
    \item Section~\ref{sec:tests-ex-3-4-2}: Test results
    \end{enumerate}
  \item Chapter~\ref{cha:exercise-3-4-2}: Exercise 3.4.2
    \begin{enumerate}[{}]
    \item Section~\ref{sec:problem-statement-ex-3-4-2}: Problem statement
    \item Section~\ref{sec:relevant-code-ex-3-4-2}: Relevant code
    \item Section~\ref{sec:tests-ex-3-4-2}: Test results
    \end{enumerate}
  \item Chapter~\ref{cha:source-code-ex-2-5-1}: Source Code for Exercise
    2.5.1
  \item Chapter~\ref{cha:source-code-ex-3-4-1}: Source Code for Exercise
    3.4.1
  \item Chapter~\ref{cha:source-code-ex-3-4-1}: Source Code for Exercise
    3.4.2
  \end{enumerate}
\item[Reproducibility in ML and \LaTeX{}] \ \\
  Our ML and \LaTeX{} source files compile with no errors.
\end{description}


\chapter{Exercise 2.5.1}
\label{cha:exercise-2-5-1}

\section{Problem Statement for Exercise 2.5.1}
\label{sec:problem-statement-ex-2-5-1}
  For this problem I solved the Exercise 2.5.1 from the PDF book.
  This problem was to create a variable called timesPlus and then
  pass variables into the equation in order to prove the code was
  correct.

\section{Relevant Code for Exercise 2.5.1}
\label{sec:relevant-code-ex-2-5-1}
  The relevant code wil be in the corresponding Appendix at the
  end of the report.

\subsection{Test cases for Exercise 2.5.1}
\label{sec:test-casesex-2-5-1}
The relevant code wil be in the corresponding Relevant Information 
Chpater at the end of the report.

\section{Test Results for Exercise 2.5.1}
\label{sec:tests-ex-2-5-1}

Below are the results from running the test cases:

The following is output from \emph{ex-2-5-1.sml}
\lstinputlisting{ML/ex-2-5-1.trans}

\chapter{Exercise 3.4.1}
\label{cha:exercise-3-4-1}

\section{Problem Statement for Exercise 3.4.1}
\label{sec:problem-statement-ex-3-4-1}
  For this problem I solved the Exercise 3.4.1 from the PDF book.
  This problem was to create a list and then using the cons operator
  gradually split up the lists into different lists. The intial list
  had 4 pairs with a index number and then a person's name. I did use
  a temp variable to split up the last values in order to continue to
  use the cons structure.

\section{Relevant Code for Exercise 3.4.1}
\label{sec:relevant-code-ex-3-4-1}
  The relevant code wil be in the corresponding Appendix at the
  end of the report.

\subsection{Test cases for Exercise 3.4.1}
\label{sec:test-cases-ex-3-4-1}
The relevant code wil be in the corresponding Relevant Information 
Chpater at the end of the report.

\section{Test Results for Exercise 3.4.1}
\label{sec:tests-ex-3-4-1}

Below are the results from running the test cases:

The following is output from \emph{ex-3-4-1.sml}
\lstinputlisting{ML/ex-3-4-1.trans}

\chapter{Exercise 3.4.2}
\label{cha:exercise-3-4-2}

\section{Problem Statement for Exercise 3.4.2}
\label{sec:problem-statement-ex-3-4-2}
  For this problem I solved the Exercise 3.4.2 from the PDF book.
  This problem was to copy the code provided in the book and then
  to run it. Once we did that in order we had to explain the findings.
  I did this in the source code (surrounded by comments so the code
  should still compile).

\section{Relevant Code for Exercise 3.4.2}
\label{sec:relevant-code-ex-3-4-2}
  The relevant code wil be in the corresponding Appendix at the
  end of the report.

\subsection{Test cases for Exercise 3.4.2}
\label{sec:test-cases-ex-3-4-2}
  The relevant code wil be in the corresponding Relevant Information 
  Chpater at the end of the report.

\section{Test Results for Exercise 3.4.2}
\label{sec:tests-ex-3-4-2}

Below are the results from running the test cases:

The following is output from \emph{ex-3-4-2.sml}
\lstinputlisting{ML/ex-3-4-2.trans}

\chapter{Relevant Information}
\label{cha:relevant-information}

\section{Specific Test Cases}
\label{sec:specific-test-cases}

\subsection{Test Cases for Exercise 2-5-1}
\label{sec:test-cases-ex-2-5-1}
The required tests from the PDF project 1 handout are below:
\begin{lstlisting}[frame=TB]
  (******************************************************************************) 
  (* Test Cases *) 
  (******************************************************************************) 
  timesPlus 100 27;
  timesPlus 10 26; 
  timesPlus 1 25; 
  timesPlus 2 24; 
  timesPlus 30 23; 
  timesPlus 50 200;
 \end{lstlisting}

\subsection{Test Cases for Exercise 3-4-1}
\label{sec:test-cases-ex-3-4-1}
The required test cases as shown in the PDF project 1 handout (and
parts a, b, and c in the textbook).
\begin{lstlisting}[frame=TB]
  (a) The list of pairs assigned to listA
  (b) The ML expression that results in the assignment of values to e1B 
  and listB as specified
  (c) The ML expressions that result in the assignment of values to elC1,
  elC2, elC3, elC4, and elC5 as specified
\end{lstlisting}

\subsection{Test Cases for Exercise 3.4.2}
\label{sec:test-cases-ex-3-4-2}

For this exercise there was no actual test cases we just had to run
the provided code and explain the output.

%% ------------------------------------------
%% this restarts the section numbering and
%% changes chapter numbering to letters starting
%% with A
%% ------------------------------------------
\appendix{} 


\chapter{Source Code for  Exercise 2.5.1}
\label{cha:source-code-exercise-2-5-1}

The following code is from \emph{ex-2-5-1.sml}
\lstinputlisting{ML/ex-2-5-1.sml}

\chapter{Source Code for  Exercise 3.4.1}
\label{cha:source-code-exercise-3-4-1}

The following code is from \emph{ex-3-4-1.sml}
\lstinputlisting{ML/ex-3-4-1.sml}

\chapter{Source Code for  Exercise 3.4.2}
\label{cha:source-code-exercise-3-4-2}

The following code is from \emph{ex-3-4-2.sml}
\lstinputlisting{ML/ex-3-4-2.sml}

\end{document}